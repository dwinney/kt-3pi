\documentclass[10pt, aps,prd,amsmath,amssymb,superscriptaddress,onecolumn,
nofootinbib,showpacs,preprintnumbers]{revtex4-1}

\bibliographystyle{apsrev4-1.bst}
\usepackage{hyperref,color,subfigure}
\usepackage{amsmath,amssymb,physics,mathtools}
\usepackage[utf8]{inputenc}
\usepackage{lmodern,bm}
\usepackage{graphicx,dcolumn,tikz,xcolor,xfrac}
\usepackage{cleveref}
\usepackage{upgreek}

\newcommand{\mand}{\qquad \text{ and } \qquad}
\newcommand{\jmax}{{j_\text{max}}}
\newcommand{\lmax}{{\ell_\text{max}}}
\newcommand{\lpmax}{{l^\prime_\text{max}}}
\newcommand{\jpmax}{{j^\prime_\text{max}}}
\newcommand{\Ip}{{I^\prime}}
\newcommand{\jp}{{j^\prime}}
\newcommand{\Disc}{\text{Disc }}
\newcommand{\emp}{{m^\prime}}
\newcommand{\lamp}{{\lambda^\prime}}

\renewcommand*{\arraystretch}{1.5}

\begin{document}
%#########################################################################################################
%#########################################################################################################
%#########################################################################################################
\begin{center}
\large \textbf{Axial-Vector Meson, \, \(I^G J^{PC} = 1^-1^{++}\)}
\end{center}
Here we look at the axial vector decaying into three pions in the Khuri-Treiman formalism, however we will try to keep formulas more general in terms of the mass of the decaying particle, \(M\), and its quantum numbers, \(I^G \, J^{PC}\). Greek indices indicate spacetime components and Latin indices represent isospin projections.
In the decay physical region of an axial-vector meson :
%%
  \begin{equation}
    \label{eq:decay-channel}
    M^d(p_M, \lambda) \rightarrow \pi_1^a(p_1)\;  \pi_2^b(p_2) \; \pi^c_3(p_3),
  \end{equation}
we define the usual invariant, Mandelstam variables
%%
  \begin{align} \label{eq:invariants}
    s = (p_M - p_3)^2,  \qquad \qquad t = (p_M - p_1)^2,  \qquad  \qquad u = (p_M-p_2)^2.
  \end{align}
%%
Using crossing symmetry, we can relate \cref{eq:decay-channel} in the physical decay region to the \(2\to2\) scattering channel by analytic continuing the energy variables. The scattering amplitude will in general depend on the isospin projection of all four particles and the helicity of the axial vector meson.

Because all the particles are isovectors, the isospin decomposition is identical to that of \(\pi\pi\to\pi\pi\), as in \cite{Albaladejo2018}:
%%
  \begin{equation}
    \mel{\pi^a(p_1)\pi^b(p_2)}{\hat{T_\lambda}}{\pi^c(\overline{p}_3)A^d(p_A)} = \delta_{ab}\delta_{cd} \; \mathcal{A}^{c}_\lambda(s,t,u) + \delta_{ad}\delta_{bc} \; \mathcal{A}^{a}_\lambda(s,t,u) + \delta_{ac}\delta_{bd} \; \mathcal{A}^{b}_\lambda(s,t,u)
  \end{equation}
%%
  here however the scalar amplitudes \(A^c_\lambda(s,t,u)\) depends on kinematic variables and additionally the helicity of the vector meson and the isospin projection of the pion in the initial state.

Because the different initial state pion projections should be related by crossing, we can relate all three scalar amplitudes in \cref{eq:scalar-decomp} to a single amplitude, \(\mathcal{A}_\lambda(s,t,u) \equiv \mathcal{A}_\lambda^{c}(s,t,u)\) by permutations of Mandelstam variables,
%%
\begin{equation}
  \label{eq:scalar-decomp}
  \mel{\pi^a(p_1)\pi^b(p_2)}{\hat{T_\lambda}}{\pi^c(\overline{p}_3)M^d(p_A)} = \delta_{ab}\delta_{cd} \; \mathcal{A}_\lambda(s,t,u) + \delta_{ad}\delta_{cb} \; \mathcal{A}_\lambda(t,s,u) + \delta_{ac}\delta_{bd} \; \mathcal{A}_\lambda(u,t,s).
\end{equation}
%%
We note that we thus only have two independent scalar amplitudes corresponding to the two unique helicity projections of the axial vector (see \cref{sec:kin-singularities}).

We can build isospin-definite scattering amplitudes by linear combinations of the \(s, \; t \text{ or } u\) channel such that
%%
	\begin{equation}\label{eq:iso-decomp}
	  \mel{\pi^a(p_1)\pi^b(p_2)}{\hat{T_\lambda}}{\pi^c(p_3)M^d(p_M)} = P^{(0)}_{abcd} \; A_\lambda^{(0)}(s,t,u) + P^{(1)}_{abcd}  \; A_\lambda^{(1)}(s,t,u) +  P^{(2)}_{abcd} \; A_\lambda^{(2)}(s,t,u)
	\end{equation}
%%
with projectors,
%%
	\begin{equation} \label{eq:projectors}
	P^{(0)}_{abcd} = \frac{1}{3}\delta_{ab}\delta_{cd},  \quad P^{(1)}_{abcd} = \frac{1}{2}(\delta_{ac}\delta_{bd}-\delta_{ad}\delta_{bc}),  \quad \textrm{and} \quad   P^{(2)}_{abcd} = \frac{1}{2}
	(\delta_{ac}\delta_{bd} + \delta_{ad}\delta_{bc}) - \frac{1}{3} \delta_{ab}\delta_{cd}.
	\end{equation}
%%
Combining\cref{eq:iso-decomp,eq:projectors} and comparing to \cref{eq:scalar-decomp}, we get
%%
    \begin{align} \label{eq:matrix}
      \begin{bmatrix}
      \mathcal{A}_\lambda^{(0)}(s,t,u) \\ \mathcal{A}_\lambda^{(1)} (s,t,u) \\ \mathcal{A}_\lambda^{(2)}(s,t,u)
      \end{bmatrix}
    =
      \begin{bmatrix*}[r]
        3 & 1 & 1 \\ 	0 & 1 & -1 \\ 0 & 1 & 1
      \end{bmatrix*}
      \begin{bmatrix}
      \mathcal{A}_\lambda(s,t,u) \\ \mathcal{A}_\lambda(t,s,u) \\ \mathcal{A}_\lambda(u,t,s)
      \end{bmatrix}.
    \end{align}
%%
 Each of these amplitudes will still depend on the helicity, giving us matrix equations that become unwieldy very fast. We wish to compare to the ``freed-isobar" partial-wave analysis results from COMPASS \cite{COMPASS-Swave,Krinner:2017vch}, which extracts partial waves with an exclusive \(\pi^-\pi^-\pi^+\) final state.
%########################################################################################################
%#########################################################################################################
%########################################################################################################
\section{Helicity Partial Wave Amplitudes} \label{sec:helicity}
Following \cite{Mikhasenko:2017rkh} We start by writing the matrix element as a sum over \(s\)-channel helicity amplitudes (see eq. 10-5 in~\cite{perl}) by considering the scattering channel \(M(p_M, \lambda) \pi_3(\overline{p}_3) \to  \pi_1(p_1)\pi_2(p_2)\):
%%
  \begin{equation}
    \label{eq:helicity}
    \mathcal{A}_\lambda(s,z_s) = \sum_{j= |\lambda|}^\infty \, (2j +1) \; e^{i\lambda\varphi} \, d_{\lambda0}^j(\theta_s) \; A_{j \lambda}(s).
  \end{equation}
%%
The angular dependence of the decay is described by Wigner-\(d\) functions of the \(s\)-channel scattering angle, \(\theta_s\), if we choose the \(x-z\) plane as the scattering plane (i.e. \(\varphi = 0 \) ).
%%
\subsection{Kinematic Singularities}
\label{sec:kin-singularities}
We wish to factor out all kinematic singularities in \(s\) and \(\theta_s\) from the helicity partial wave and rotational function respectively. First we define the kinematic-free \(d\)-function, denoted with a hat, such that:
%%
  \begin{equation}
      \label{eq:halfangle}
      d^j_{\lambda 0}(\theta_s) = \hat{d}^j_{\lambda 0}(\theta_s) \; \xi_{\lambda}(z_s)
       \quad \text{ where } \quad
      \xi_{\lambda}(z_s) = \bigg( \sqrt{ 1- z_s^2} \bigg)^{|\lambda|}
       \quad \text{ and } \quad
      \hat{d}^j_{\lambda 0}(\theta_s) = \frac{d^{|\lambda|}}{d \, z_s^{|\lambda|}} (P_j(z_s))
  \end{equation}
%%
where \(\xi_{\lambda}\) are the ``half-angle factors" (see eq. 4.4.12 in \cite{Collins}).

We similarly factor out singularities in \(s\) by defining:
%%
  \begin{equation}
    \label{eq:kinematicfreepartialwave}
    A^{(s)}_{j\lambda}(s) = s^{-|\lambda|/2} \;  (k(s)q(s))^{j - |\lambda|} \, \tilde{K}_{\lambda} \times \hat{A}_{j\lambda}(s).
  \end{equation}
%%
Here
  \begin{equation}
    \label{eq:momenta}
    q(s) = \frac{\lambda^{1/2}(m_\pi^2, m_\pi^2, s)}{2\sqrt{s}} = \frac{\lambda_\pi^{1/2}(s)}{2\sqrt{s}}
     \qquad \text{ and } \qquad
     k(s) = \frac{\lambda^{1/2}(M^2, m_\pi^2, s)}{2\sqrt{s}} = \frac{\lambda_M^{1/2}(s)}{2\sqrt{s}} ,
  \end{equation}
with \(\lambda(x,y,z) = x^2 + y^2 + z^2 - 2 (xy + yz + zx)\) is the K\"{a}ll\'{e}n function, are the magnitudes of  the relative momentum between outgoing pions and the incoming pion's momentum respectively. \(M\) is the mass of the decaying particle. The \((kq)^{j-\lambda}\) term is included to cancel out singularities in \(s\) from the \(d\)-function at threshold and pseudo-threshold (see eq. 6.2.9 in \cite{Collins}).

The other kinematic factor, \(\tilde{K}_{\lambda}\) arise  because near threshold \(\hat{A}_{j\lambda}(s)\) has additional power behavior of \(k(s)\) or \(q(s)\) corresponding to the dependence on \(j\) and \(L\) between helicity amplitudes and \(LS\) amplitudes (see discussions in \cite{Jackson1968,Franklin1966}). Near threshold:
%%
  \begin{gather}
    A_{j\lambda}(s) \sim k(s)^{L_i} \, q(s)^{L_f} \times (k(s)q(s))^{j- |\lambda|}
  \end{gather}
%%
where \(L_i\) and \(L_f\) are the minimum angular momentum of the initial and final states for the given helicity. We have:
%%
  \begin{align}
      L_i = 1, L_f = 0 &\qquad  \text{ for } \lambda = 0, \; j = 0   \nonumber \\
      L_i = -1, L_f = 0 &\qquad  \text{ for } \lambda = 0, \;  j \label{eq:lilf} \geq 1  \\
      L_i = 0 , L_f = 1 & \qquad  \text{ for } \lambda = 1, \; j \geq 0 \nonumber
  \end{align}
%%
We note that we must treat the \(j=0\) term in \cref{eq:kinematicfreepartialwave} differently, since it has a different lowest orbital angular momentum possible than \(j\geq 1\). The explicit form for \(\tilde{K}_{\lambda}\) is given by Table 6.1 in \cite{Collins}. For the axial-vector decay:
%%
  \begin{align}
    \label{eq:k-factor-tilde}
    \tilde{K}_{\lambda}(s) =
    \begin{cases}
       \lambda^{1/2}_M(s)    &     \text{  for }\lambda = 0, \; j=0 \\
       \lambda^{-1/2}_M(s)  &\text{ for } \lambda = 0, \; j\geq 1 \\
       \lambda_\pi^{1/2}(s)     &\text{ for } \lambda = 1, \; j\geq 0
    \end{cases}
  \end{align}
%%
which is consistent with \cref{eq:lilf}. In general, for a particle with quantum numbers \(J^{PC}\) decaying with helicity \(\lambda\) into three pions has
%%
  \begin{align}
    \tilde{K}_{\lambda }(s) &=  (2 \; \sqrt{s} \; k(s))^{|\lambda| + Y_i} \; (2 \; \sqrt{s} \; q(s))^{|\lambda| + Y_f} \nonumber \\
    &=   \big(\lambda^{1/2}_M(s)\big)^{|\lambda| + Y_i} \; \big(\lambda_\pi^{1/2}(s)\big)^{|\lambda| + Y_f}\, .
  \end{align}
%5
with
%%
  \begin{equation}
    Y_i = - J \mand Y_f =  \frac{1}{2} \big[ 1 + P (-1)^J]
    \end{equation}
%%
We note that the kinematic singularity functions are related to the Lorentz-covariant Kibble function, \(\phi\), by:
%%
  \begin{equation}
    \label{eq:k-factor}
    K_\lambda(s) = s^{-|\lambda|/2} \; \xi_{\lambda}(z_s) \; \tilde{K}_{\lambda}(s) = \sqrt{\phi^{|\lambda|}} \; \sqrt{\lambda_M^{Y_i}(s) \; \lambda_\pi^{Y_f}(s)}\;  .
  \end{equation}
%%
Explicitly, the Kibble function can be expressed in terms of invariants as:
%%
  \begin{equation}
    \phi = (2\sqrt{s} \; \sin \theta_s \; k(s) q(s))^2 =  \bigg( \sin\theta_s \; \frac{\lambda^{1/2}_M(s) \; \lambda^{1/2}_\pi(s)}{2 \,\sqrt{s}} \bigg)^2 = s \, t\, u - m_\pi^2 \;(M^2 - m_\pi^2)^2 \; ,
  \end{equation}
%%
with \(\phi = 0\) defining the boundaries of physical kinematic regions.

The fully factorized amplitudes in terms of kinematic-singularity-free helicity amplitudes then takes the form
%%
  \begin{equation}
    \label{eq:helicity-final}
    \mathcal{A}_\lambda(s,z_s) =  K_{\lambda}(s)  \sum_{j= |\lambda|}^\infty (2j+1) \, (k(s)q(s))^{j- |\lambda|} \; \hat{d}^j_{\lambda0}(\theta_s)  \; \hat{A}_{j\lambda}(s)
  \end{equation}
%%
with care to treat the \(\lambda =0, \;  j=0\) case as discussed above in \cref{eq:lilf}.
%#########################################################################################################
%#########################################################################################################
%#########################################################################################################
\subsection{Parity and Bose Symmetry} \label{sec:symmetry}
In general we have \((2J+1)\) helicity amplitudes to consider but we can use parity to reduce this number.

The amplitude must conserve parity, so since the axial-vector meson is even under parity, \( P_A = +1\), and using \(P_\pi = -1\), we can relate \(A_{j+}(s) = A_{j-}(s)\). Or more generally:
%%
  \begin{equation}
    \label{parity-jpc}
    \hat{A}_{-\lambda}(s,z_s) = -P(-1)^{J+\lambda} \; \hat{A}_{\lambda}(s,z_s)
  \end{equation}
%%

In general, for parity even mesons such as the \(a_1(1260)\), we will have \((J +1)\) helicity amplitudes and parity odd (\(\omega \text{ or } \phi\)) will have \(J\) helicity amplitudes because \(\hat{A}_{j0}(s) = 0\). So for our axial vector we only have two independent helicity amplitudes and we choose to consider \(\lambda = 0,\; +1\).

We rewrite \cref{eq:helicity}
%%
  \begin{align}
    \label{eq:model-helicity-zero}
    \mathcal{A}_0(s,z_s) &= \frac{1}{K_{0}(s)} \, \hat{A}_{00}(s) + K_{0}(s) \sum_{j = 0}^\infty (2j+1) \, (k(s)q(s))^j \; \hat{d}_{00}^j(\theta_s) \, \hat{A}_{j0}(s) \\
    \nonumber \\
    \label{eq:model-helicity-plus}
    \mathcal{A}_+(s,z_s) &=  K_{+}(s) \sum_{j = 2}^\infty (2j+1) \ (k(s)q(s))^{j-1} \;  \hat{d}_{10}^j(\theta_s) \, \hat{A}_{j+}(s) \,.
   \end{align}
%%
We recall that \(\hat{d}^j_{\lambda 0}(\theta_s)\) and \(\hat{A}_{j\lambda}(s)\) are free of any kinematic singularities and the analytic structure therefore is solely dynamical.

The sums can be further restricted by imposing Bose symmetry. Because we have identical pions in the final state, the amplitude must be invariant under the interchange \(z_s \to -z_s\) regardless of the helicity of the decaying meson.
Because \(d^j_{\lambda0}(\theta_s) \propto P^\lambda_j(z_s)\) where these are the associated Legendre polynomials, the amplitudes transform as
%%
  \begin{equation}
    \label{eq:iso-bose}
   \mathcal{A}_{\lambda}(s, -z_s) \propto \sum_{j} (-1)^{j+\lambda} \; \times (2j+1) \;  d^j_{\lambda0}(\theta_s) \; A_{j\lambda}(s) \,.
  \end{equation}
%%
The effect of interchanging pions is discussed in \cite{JACOB1959404}, and should be independent of the helicity of the decaying particle in the initial state.
%%
  \begin{equation}
    \label{eq:pipi-bose}
    \ket{j \; m ; \pi_1(p_1) \, \pi_2(p_2)} = (-1)^j \; \ket{j \; m; \pi_1(p_2) \, \pi_2(p_1)}.
  \end{equation}
%%
Examining \cref{eq:iso-bose}, the only way for \cref{eq:pipi-bose} to be satisfied in general is if:
%%
  \begin{equation}
    \label{eq:helicity-bose}
    \mathcal{A}_{\lambda}(s,-z_s) = (-1)^\lambda \; \mathcal{A}_{\lambda}(s, z_s),
  \end{equation}
%%
this is the Bose symmetry condition for a meson decaying into three pions, (without definite isospin).
Thus, \(\hat{A}_{j\lambda}(s) = 0\) unless \(j\) is even and so we can restrict the sum to even values.

If we define isospin-definite amplitudes by \cref{eq:matrix}, the analogous expression to \cref{eq:helicity-bose} gives
%%
  \begin{equation}
    \label{eq:helicity-iso-bose}
      \mathcal{A}^{(I)}_{\lambda}(s,-z_s) = (-1)^{I + \lambda} \; \mathcal{A}^{(I)}_{\lambda}(s, z_s) \;,
  \end{equation}
%%
since the \(\pi\pi\) final state in a configuration of definite isospin-\(I\) transforms as
%%
  \begin{equation}
    \label{eq:pipi-iso-bose}
    \ket{j \; m, \, I ;  \pi^a_1(p_1) \, \pi^b_2(p_2)} = (-1)^{j+I} \, \ket{j \; m, \, I; \pi^b_1(p_2) \, \pi^a_2(p_1)}
    \; .
  \end{equation}
%%
Thus the sums over isospin-definite, helicity partial waves is restricted to values of \(j\) with \(I + j\) is even.
 %#######################################)#################################################################
 %#########################################################################################################
 %#########################################################################################################
 \subsection{Connection to Covariant Scalar Amplitudes} \label{sec:covariant}
 In this section we relate the helicity amplitudes of the previous section to Lorentz scalar amplitudes. Writing amplitudes in terms of covariant structures is in general more process dependent and obtaining equations for the discontinuities is more challenging. We illustrate the procedure with the axial vector meson in order to match with previous results derived directly from the helicity amplitudes.

   We start by writing out the most general covariant structure, contracting the polarization tensor of the decaying mesons with two independent combinations of the momenta of the pions:
 %%
   \begin{equation}
     \label{eq:covariant}
     \mathcal{A}_\lambda(s,t,u) = \epsilon_\mu^\lambda(p_M) \, \bigg[ F(s,t,u) \; (p_1 + p_2)^\mu + G(s,t,u) \;  (p_1 - p_2)^\mu \bigg].
     \end{equation}
 %%
 where \(F\) and \(G\) are two independent, Lorentz-scalar amplitudes. The momenta of the pions are \(p_i\), and the axial vector meson, \(M\), has polarization vector \(\epsilon_\mu\) which depends on helicity, \(\lambda\), energy in the center of mass frame, \(E_M\), and momentum \(p_M\):
 %%
   \begin{equation}
     \label{eq:polarization}
     \epsilon_\mu(p_M, \pm1) = \frac{1}{\sqrt{2}} \big( 0, \mp 1, - i, 0 \big) \qquad \text{ and } \qquad \epsilon_\mu(p_M, 0) = \frac{1}{M} \big( p_M, 0, 0, E_M \big).
     \end{equation}
 %%
 We also have
 %%
   \begin{gather}
     \vec{p}_1 = q(s) \; (\sin \theta_s, 0,  -\cos \theta_s) \qquad \qquad \vec{p}_2 = q(s) \; (-\sin \theta_s, 0 , \cos \theta_s ) \\
     \vec{p}_3 = - \vec{p}_M = k(s) \; (0,0,-1). \nonumber
   \end{gather}
 %%

 The  tensor structure in \cref{eq:covariant} highlights the intrinsic Bose symmetry of the reaction, i.e. because we have identical pions in the final states, the total amplitude should be invariant under the the interchange \(t \leftrightarrow u \) or \(p_1 \leftrightarrow p_2\), giving us the relations:
 %%
   \begin{equation}
     F(s,t,u) = F(s,u,t) \qquad \text{ and } \qquad G(s,t,u) = - \, G(s,u,t).
   \end{equation}
 %%

 We reiterate that the expression \cref{eq:covariant} defines the helicity, \(\lambda\), in the \(s\)-channel scattering frame, meaning the spin is quantized in the direction of \(p_3\), in the \(M \, \pi_3\) center-of-mass frame.
 In this frame, we use \cref{eq:momenta,eq:polarization} for \(\lambda = 0 \text{ and } +1 \):
 %%
   \begin{align}
     \label{eq:contract_zero}
     \epsilon_\mu(p_M,0) \; p_1^\mu &= \frac{E_1}{M} \, k(s) + \frac{E_M}{M} \,  z_s \, q(s)
     \qquad \qquad \epsilon_\mu(p_M,+1) \; p_1^\mu = - \frac{1}{\sqrt{2}} \, \sqrt{1 - z_s^2} \; q(s) \nonumber \\
 %%
     \epsilon_\mu(p_M,0) \; p_2^\mu &= \frac{E_2}{M} \, k(s) - \frac{E_M}{M} \,  z_s \, q(s)
     \qquad \qquad \epsilon_\mu(p_M,+1) \; p_2^\mu =  \frac{1}{\sqrt{2}} \, \sqrt{1 - z_s^2} \; q(s)  \\
 %%
     \epsilon_\mu(p_M,0) \; p_3^\mu &= \frac{k(s)}{M} \, \sqrt{s}
     \qquad \qquad \qquad \quad \epsilon_\mu(p_M,+1) \; p_3^\mu = 0 \nonumber
   \end{align}
 %%
 We also have
 %%
   \begin{equation}
     \label{eq:energies}
     E_1 = E_2 = \frac{\sqrt{s}}{2} \mand E_M = \frac{s + M^2 - m_\pi^2}{2 \sqrt{s}}
   \end{equation}
 Using \cref{eq:contract_zero} in \cref{eq:covariant}, we get the helicity amplitude in terms of its covariant form factors to directly compare with \cref{eq:helicity}:
 %%
  \begin{align}
   \label{eq:covariant_zero}
    \mathcal{A}_0 =& \; \frac{\sqrt{s}}{M} \, k(s) \; F(s,t,u) + 2 \; \frac{E_M}{M} \, z_s \; q(s) \; G(s,t,u) \\
    \nonumber \\
    \label{eq:covariant_plus}
    \mathcal{A}_+ =& \; - \sqrt{2} \sqrt{1-z_s^2} \, q(s) \; G(s,t,u).
  \end{align}
 %%
 Comparing \cref{eq:covariant_zero} with \cref{eq:helicity} and using \cref{eq:halfangle,eq:k-factor}, we can match:
 %%
   \begin{equation}
     \label{eq:matching_G}
     G(s,t,u) =  - \sqrt{2} \; \sum_{j = 0}^\infty (2j+1) \, (k(s)q(s))^{j - 1} \,\hat{d}^j_{10}(\theta_s) \, \hat{A}_{j+}(s)
   \end{equation}
 %%
 Matching the other helicity amplitude and using \cref{eq:matching_G},
 %%
   \begin{align}
     \label{eq:matching_F}
     F(s,t,u) &= 2 M \, \hat{A}_{00}(s) \nonumber
     \\
     &+ \, \frac{2 \; M}{\lambda_M(s)} \times \sum_{j \not= 0}^\infty (2j+1) \, (k(s)\,q(s))^{j} \;
     \bigg[
        \hat{d}^j_{00}(\theta_s) \; \hat{A}_{j0}(s)
      %%
   +  2 \sqrt{2} \; \frac{(s + M^2 - m_\pi^2)}{M}\;  \,  z_s \, \hat{d}^j_{10}(\theta_s) \; \hat{A}_{j+}(s)
   \bigg]
   \end{align}
 %%
 Here we recall that \(\hat{A}_{j+}\) and \(\hat{A}_{j0}\) vanish for odd values of \(j\). Additionally since we want \(F\) and \(G\) to be free of any kinematic singularities, the \(\lambda_M^{-1}\) term in front of the sum for \(j\geq 1\) poses a problem as it adds two poles at the pseudo-thresholds: \(s = (M \pm m_\pi)^2\).
 This means our helicity amplitudes are not completely independent of each other at the pseudo-threshold as the terms in the brackets in \cref{eq:matching_F} must vanish at these points as discussed in \cite{Mikhasenko:2017rkh}. In other words, at these points, we have the kinematic constraint between the two helicity partial waves as \(s \to (M \pm m_\pi)^2\).
 %%
 \begin{align}
   \label{constraint-invariant}
  \bigg[  \hat{d}_{00}^j(\theta_s) \; \hat{A}_{j0}(s) + 2 \; \sqrt{2} \; \frac{(s+ M^2 - m_\pi^2)}{M} \; z_s \; \hat{d}^j_{10}(\theta_s) \; \hat{A}_{j+}(s) \bigg]
   \to 0 \text{ as } \lambda_M(s) \; .
 \end{align}
 %%
%#########################################################################################################
%#########################################################################################################
%#########################################################################################################
\subsection{Crossing Symmetry}
 So far we have considered only \(s\)-channel helicity partial waves, but equivalently we may write an analogous expansion to \cref{eq:helicity} in the \(t\)-channel. In other words, denoting the helicity projection with a superscript:
 %%
 \begin{equation}
   \label{eq:helicity-t}
   \mathcal{A}^{(t)}_\lambda(s, t, u) = \sum_{j = |\lambda|}^\infty \, (2j+1) \; d^{j}_{\lambda0}(\theta_t) \; A^{(t)}_{j\lambda}(t) \; .
 \end{equation}
%%%
Both infinite sum expansions are exact and describe the same amplitude, in terms of different orientations of the helicity. The \(t\)-channel amplitudes of \cref{eq:helicity-t} then are related to the \(s\)-channel helicity amplitudes \cref{eq:helicity} by a Wigner rotation that takes one orientation of \(\lambda\) to the other.
In terms of the \(s\)-channel scattering channel considered in \cref{sec:helicity} with \(\lambda\) in the direction of \(-\bar{p}_3\) and \cref{eq:invariants}, we write the crossing symmetry relation:
%%
  \begin{equation}
    \label{eq:crossing-relation}
    \mathcal{A}_\lambda^{(t)}(s, t, u) = \sum_{\lambda^\prime} \; d_{\lambda^\prime \lambda}^J(\hat{\theta}_1) \; \mathcal{A}_{\lambda^\prime}^{(s)}(s,t,u)
  \end{equation}
%%
where \(\hat{\theta}_1\) the angle between \(p_3\) and \(p_1\) in the total CM frame \((p_M = 0\)). We can analogously define \(u\)-channel helicity partial waves with \(\hat{\theta}_2\), the angle between \(p_3\) and \(p_2\):
%%
  \begin{align} \label{sin-hat}
    \sin\hat{\theta}_1 = \frac{
    2 \; M \; \sqrt{\phi}
    }{
    \sqrt{\lambda_M(s) \;  \lambda_M(t)}
    }
    \mand
    \sin\hat{\theta}_2 = \frac{
    2 \; M \; \sqrt{\phi}
    }{
    \sqrt{\lambda_M(s) \;  \lambda_M(u)}
    }
  \end{align}
%%
or
%%
  \begin{align} \label{cos-hat}
    \cos\hat{\theta}_1 &= \frac{n(s,t)}
    {\sqrt{\lambda_M(s) \; \lambda_M(t)}} \mand
    \cos\hat{\theta}_2 = \frac{n(s,u)}
    {\sqrt{\lambda_M(s) \; \lambda_M(u)}} \; .
  \end{align}
%%
for \(n(s,t) = (M^2 + m_\pi^2 - s)(M^2 + m_\pi^2 - t) + 2 \; M^2 (2m_\pi^2 - u)\) is polynomial and symmetric in \(s\) and \(t\).

Using the symmetry properties discussed in \cref{sec:symmetry}, we may rewrite the right-hand side of \cref{eq:crossing-relation} for a general \(J^{PC}\) decaying with helicity \(\lambda\) in the \(s\)-channel, with \(\mu > 0\):
%%
  \begin{align} \label{gggg}
     \mathcal{A}_\lambda^{(t)}(s, t, u) = \frac{1}{2} \big[1- P(-1)^J \big] \; (-1)^\lambda \; d_{ \lambda0}^J(\hat{\theta}_1) \; \mathcal{A}^{(s)}_0(s,t,u)
       + \sum_{\lamp = 1}^J \bigg[ d_{ \lamp \lambda}^J(\hat{\theta}_1) - P (-1)^{J+\lambda} \; d_{-\lamp, \, \lambda }^J(\hat{\theta}_1) \bigg] \; \mathcal{A}_{\lamp}^{(s)}(s,t,u)
  \end{align}
%%
or specifically for the axial vector meson,
%%
  \begin{align} \label{matrix-helicity}
    \begin{bmatrix}
  \mathcal{A}^{(t)}_0(s,t,u) \\
   \mathcal{A}^{(t)}_+(s,t,u)
    \end{bmatrix}
    =
    \begin{bmatrix}
       \cos \hat{\theta}_1   &  - \sqrt{2} \; \sin \hat{\theta}_1   \\
       \sin \hat{\theta}_1 / \sqrt{2}   &  \cos \hat{\theta}_1
    \end{bmatrix}
    \times
    \begin{bmatrix}
  \mathcal{A}^{(s)}_0(s,t,u) \\
   \mathcal{A}^{(s)}_+(s,t,u)
    \end{bmatrix}
  \end{align}
%%
where we have made use of the identities:
%%
  \begin{equation}
    d_{\lambda \lamp}^J(\hat{\theta}_1) = d^J_{-\lamp, -\lambda}(\hat{\theta}_1) = (-1)^{\lambda - \lamp} d_{\lamp \lambda}(\hat{\theta}_1) \; .
  \end{equation}
%%
Following the procedure of \cref{sec:kin-singularities} we seperate the kinematic singularities of \(\mathcal{A}_\lambda^{(s)}\). We define the kinematic-singularity-free sum
\(\hat{a}_\lambda^{(s)} = \sum_j (2j+1) \; (k(s)q(s))^{j-|\lambda|} \; \hat{d}^j_{\lambda0}(\theta_s) \; \hat{a}^{(s)}_{j\lambda}(s)\) to alleviate the notation and focus on the relation between different helicity amplitudes.
%%
  \begin{align} \label{matrix-helicity}
    \begin{bmatrix}
        K_{0}(t)   & 0 \\
        0  &  K_{+}(t)
    \end{bmatrix}
    \begin{bmatrix}
      \hat{a}^{(t)}_{0}(t) \\
      \hat{a}^{(t)}_{+}(t)
    \end{bmatrix}
    =
    \begin{bmatrix}
      \cos \hat{\theta}_1   &  - \sqrt{2} \; \sin \hat{\theta}_1   \\
       \sin \hat{\theta}_1 / \sqrt{2}  &  \cos \hat{\theta}_1
    \end{bmatrix}
    \begin{bmatrix}
        K_{0}(s)   & 0 \\
        0  &  K_{+}(s)
    \end{bmatrix}
    \times
    \begin{bmatrix}
  \hat{a}^{(s)}_{0}(s) \\
  \hat{a}^{(s)}_{+}(s)
    \end{bmatrix}
  \end{align}
%%
Inserting \cref{eq:k-factor,sin-hat,cos-hat} into the above matrix we find:
%%
  \begin{align} \label{nosing-matrix}
    \begin{bmatrix}
  \hat{a}^{(t)}_{0}(t) \\
  \sqrt{2} \; \hat{a}^{(t)}_{+}(t)
    \end{bmatrix}
    =
    \frac{1}{\lambda_M(s)}
    \begin{bmatrix}
        n(s,t)   & - 4 \; M \; \phi \\
        M  & n(s,t)
    \end{bmatrix}
    \times
    \begin{bmatrix}
  \hat{a}^{(s)}_{0}(s) \\
  \sqrt{2} \;  \hat{a}^{(s)}_{+}(s)
    \end{bmatrix} \; .
  \end{align}
%%
We see that the matrix in \ref{nosing-matrix} is completely regular with no singularities and so are \(\hat{a}_\lambda^{(x)}(x)\). Thus to cancel out the pole arising thresholds from \(\lambda_M(s)\), we need everything on the right-hand-side to be proportional to \(\lambda_M(s)\). We note this is consistent with the kinematic constraint \cref{constraint-invariant} from the scalar amplitude approach.
%#######################################)#################################################################
%#########################################################################################################
%#########################################################################################################
\subsection{Kinematic Constraints and Transversity Amplitudes} \label{sec:transetivity}
So far we have derived the kinematic constraints for the kinematic-singularity-free helicity amplitudes of the axial vector meson decay to be regular at (pseudo-)threshold using two methods: from the crossing matrix of helicity amplitudes and from their relation to Lorentz scalar amplitudes. Here we take a brief aside into transveristy amplitudes as discussed in \cite{Kotanski1968,Cohen-Tannoudji1968,McKerrell1968} which provide a way to easily generalize the kinematic constraint on helicity amplitudes, \cref{constraint-invariant}, to a decaying particle with arbitrary quantum numbers.

The transversity, \(\tau\), of a particle is spin of that particle quantized normal to the scattering plane. Because three of our particles are identical and have no spin, the relation between the transversity amplitude \(\mathcal{M}_\tau(s,t,u)\) and the helicity amplitudes as defined in \cref{sec:helicity} is simple:
%%
  \begin{equation} \label{trans-rot}
    \mathcal{M}_\tau (s,t,u) = \sum_{\lambda^\prime} D_{\lamp \tau}^{J*}\bigg(\frac{\pi}{2}, \frac{\pi}{2}, - \frac{\pi}{2} \bigg ) \times \frac{\pi}{2}\mathcal{A}_\lamp(s,t,u)
    = \sum_{\lamp} e^{\frac{i\pi}{2} (\lamp - \tau)} \; d^J_{\lamp \tau}\bigg(\frac{\pi}{2}\bigg) \times \mathcal{A}_\lamp(s,t,u) \; .
  \end{equation}
%%
The rotation in \cref{trans-rot} has the property of diagonalizing the crossing matrix \cref{eq:crossing-relation} such that the crossing relation for transversity amplitudes is
%%
  \begin{equation}
    \mathcal{M}^{(t)}_{-\tau}(s,t,u) = (-1)^J \; e^{i \, \tau \, \hat{\theta}_1} \; \mathcal{M}^{(s)}_\tau(s,t,u)
    \qquad \text{ with} \qquad
    \mathcal{M}_{-\tau}(s,t,u) = - P(-1)^{J+\tau} \; \mathcal{M}_\tau(s,t,u)
  \end{equation}
%%
where \(\hat{\theta}_1\) is the same as \cref{sin-hat} and, unlike \cref{eq:crossing-relation}, there is no sum over transversities.
The transversity amplitudes have well defined behavior near initial-state threshold branch-points at \(s= (M\pm m_\pi)^2\) (c.f. eq. (2.5) in \cite{Kotanski1968}):
%%
  \begin{equation} \label{trans-constraint}
    \mathcal{M}_\tau(s,t,u) \sim \sqrt{\lambda_M^{J + \varepsilon \, \tau}(s)}\; .
  \end{equation}
%%
Here the factor \(\varepsilon = \pm 1\) depends on the determination of the Kibble function in \cref{sin-hat} which can be shown to determine whether the behavior of the \(\mathcal{M}_\tau\) is a pole or a zero near threshold. Either choice of determination can be shown to give the same constraint on helicity amplitudes (see for example remarks below eq. (IV-11) in \cite{Cohen-Tannoudji1968}).

Thus combining \cref{trans-rot,trans-constraint} we have in general \((J + 1)\) additional kinematic constraints for our helicity amplitudes, for \(0 \leq \tau \leq J\):
%%
  \begin{equation}
    \sum_{\lamp} e^{\frac{i\pi}{2} (\lamp - \tau)} \; d^J_{\lamp \tau}\bigg(\frac{\pi}{2} \bigg) \times \mathcal{A}_\lamp(s,t,u) \sim \sqrt{\lambda^{J + \tau}_M(s)}   \; .
  \end{equation}
%%

For the axial vector, we only have one nontrivial constraint for \(\tau = +1\), at \(s = (M+m_\pi)^2\)
%%
  \begin{equation}
      \bigg[ \sqrt{2} \; \mathcal{A}_+(s,t,u) + i \; \mathcal{A}_0(s,t,u) \bigg] \to 0 \text{ as } \lambda_M(s)
  \end{equation}
%%
which we see is consistent with \cref{constraint-invariant,nosing-matrix}.
%#######################################)#################################################################
%#########################################################################################################
%#########################################################################################################
\section{Isobar Decomposition} \label{sec:isobar-decomp}
In order to evaluate the helicity amplitudes, use the isobar approximation to truncate the infinite sum in \cref{eq:model-helicity-zero,eq:model-helicity-plus}. To recover some of the high-energy singularity structure we add ``isobar helicity amplitudes" in the cross channel
%%
  \begin{equation}
    \label{eq:isobar-def}
    \mathcal{A}_\lambda(s,t,u) = A_\lambda^{(s)}(s,t,u) +  A_\lambda^{(t)}(s,t,u) + A_\lambda^{(u)}(s,t,u) \; .
  \end{equation}
%%
Here each \(A_\lambda^{(x)}\) represents a finite sum of isobar partial waves in the specified scattering channel.

 Because we are considering the decay of a particle with spin, there is additional complications of the helicity being defined in different channels. Evaluating everything in the \(s\)-channel center of mass frame, \(A_\lambda^{(s)}\) is analogous  to \cref{eq:helicity-final} with the sum truncated at some \(j_\text{max}\), however the \(t\) and \(u\) channel isobars introduce additional rotational functions.
 In general we may write (for now ignoring kinematic singularities):
 %%
 \begin{align}
   \label{eq:iso-D-matrix}
    \mathcal{A}_\lambda(s,t,u) &= \sum_{m} D^{J^*}_{\lambda m}(r_3)
    \bigg [
    \sum_{j = 0}^{\jmax} (2j+1) \; d_{\lambda0}^j(\theta_s) \; a_{ j m}^{(s)}(s)
    \bigg ]
     \nonumber \\
    &+ \sum_{m} D^{J^*}_{\lambda m}(r_1) \;
    \bigg[
    \sum_{j = 0}^{\jmax} (2j+1) \;d_{m0}^j (\theta_t) \; a_{j m}^{(t)}(t)
    \bigg]
    + \sum_{m} D^{J^*}_{\lambda m}(r_2) \;
    \bigg[
    \sum_{j = 0}^{\jmax} (-1)^{j + m} \; (2j+1) \;d_{m0}^j(\theta_u) \; a_{j m}^{(u)}(u)
    \bigg] \; .
 \end{align}
 %%
 The \(D\)-matrix arguments, \(r_i\), denote a rotation by a set of Euler angles, \((\phi_i,\theta_i,\psi_i)\), that take the \(A \, \pi_3\) CM frame (the \(s\)-channel helicity frame) to the \(A \, \pi_i\) CM frame. Since the scattering process is planar, we can in general choose \(\phi_i, \,\psi_i = 0\), with the only transformations between the different helicity frames are boosts and rotation by \(\theta_i\) to align \(p_i\) with the \(z\)-direction (opposite \(p_A\)). We note that the additional Wigner rotation in the \(s\)-channel isobars vanishes because \(\hat{\theta_3} \equiv 0\)
 in the \(s\)-channel helicity frame.

We can then rewrite \cref{eq:iso-D-matrix} as
%%
\begin{align}
  \label{eq:iso-d-func}
   \mathcal{A}_\lambda(s,t,u) &= \sum_{j = 0}^{\jmax} (2 j+1) \; d_{\lambda0}^j(\theta_s) \; a_{j \lambda}^{(s)}(s)
    \nonumber \\
   &+ \sum_{m} \sum_{j = 0}^{\jmax} (2j+1) \;
    d^{J}_{\lambda m}(\hat{\theta}_1)
    \;d_{m0}^j(\theta_t) \; a_{j m}^{(t)}(t)
    %%
   + \sum_{m} \sum_{j = 0}^{\jmax} (-1)^{j + \lambda + m} \; (2j+1) \;
    d^{J}_{\lambda m}(\hat{\theta}_2)
    \;d_{m0}^j(\theta_u) \; a_{j m}^{(u)}(u) \; ,
  \end{align}
%%
where \(\hat{\theta}_i\) the angle between \(p_3\) and \(p_i\) in the total CM frame \((p_M = 0\)):
%%
  \begin{align}
    \sin\hat{\theta}_1 = \frac{
    2 \; M \; \sqrt{\phi}
    }{
    \sqrt{\lambda_M(s) \;  \lambda_M(t)}
    }
    \mand
    \sin\hat{\theta}_2 = \frac{
    2 \; M \; \sqrt{\phi}
    }{
    \sqrt{\lambda_M(s) \;  \lambda_M(u)}
    }
  \end{align}
%%
or
%%
  \begin{align}
    \cos\hat{\theta}_1 &= \frac{n(s,t)}
    {\sqrt{\lambda_M(s) \; \lambda_M(t)}} \mand
    \cos\hat{\theta}_2 = \frac{n(s,u)}
    {\sqrt{\lambda_M(s) \; \lambda_M(u)}} \; .
  \end{align}
%%
for \(n(s,t) = (M^2 + m_\pi^2 - s)(M^2 + m_\pi^2 - t) + 2 \; M^2 (2m_\pi^2 - u)\).

 We note that the functions \(a^{(x)}_{jm}(x)\) are not the same as the helicity partial waves of the previous section. The isobar partial wave amplitudes describe an isobar intermediate state with total angular momentum \(j\), but taking the partial wave projection of \cref{eq:iso-d-func} we see we have contributions from the cross channel isobar functions as well, which we did not have in \cref{sec:helicity}.

Particles with unnatural parity which decay into \(3\pi\) correspond to isovector particles, thus we need to take into account the different isospin projections in which the reaction can resonate.
We define isospin definite isobars by plugging \cref{eq:iso-d-func} into \cref{eq:matrix} and identifying the \(s\) dependent piece:
%%
\begin{align}
  \begin{bmatrix}
  a^{(0)}_{j\lambda}(x) \\ a^{(1)}_{j\lambda}(x) \\ a^{(2)}_{j\lambda}(x)
  \end{bmatrix}
=
  \begin{bmatrix*}[r]
    3 & 1 & 1 \\ 	0 & 1 & -1 \\ 0 & 1 & 1
  \end{bmatrix*}
  \begin{bmatrix}
  a^{(s)}_{j\lambda}(x) \\ a^{(t)}_{j\lambda}(x) \\ a^{(u)}_{j\lambda}(x)
  \end{bmatrix}.
\end{align}
%%
We note this is the same definition as \cite{Albaladejo2018}, which did not have complication of spin in the \(J^P = 0^-\) case. This is because the interchanges of Mandelstam variables are related to the interchanges of particles by
%%
  \begin{equation}
    \label{frame-change}
    s\leftrightarrow  t  \qquad \pi_3 \leftrightarrow \pi_1 \qquad \Rightarrow \qquad \hat{\theta}_1 \to \hat{\theta}_3 = 0 \; ,
  \end{equation}
%%
and thus the sum over helicities for the \(t\)-channel isobars vanishes, \(d_{\lambda m}^J(0) = \delta_{\lambda m} \).

We can then relate \(\mathcal{A}_\lambda(s,t,u)\) to the isobar amplitudes of definite isospin:
%%
  \begin{align}
    \label{eq:tot-from-iso}
    \mathcal{A}_\lambda(s,t,u) &= \sum_{j = 0}^\jmax
      (2j + 1) \; d_{\lambda 0}^j(\theta_s) \;
      \frac{
     a_{j \lambda}^{(0)}(s) - a_{j \lambda}^{(2)}(s)
     }{
     3
     } \nonumber \\
     &+ \sum_m \sum_{j=0}^\jmax \; (2j+1)
     \; d^{J}_{\lambda m}(\hat{\theta}_1) \; d_{m0}^j(\theta_t) \;
     \frac{
     a^{(1)}_{j m}(t) + a^{(2)}_{j m}(t)
     }{
     2
     } \\
     &+ \sum_m \sum_{j=0}^\jmax \; (-1)^{j + m + \lambda} \; (2j+1)
     \; d^{J}_{\lambda m}(\hat{\theta}_2) \; d_{m0}^j(\theta_u) \;
     \frac{
     a^{(1)}_{jm}(u) + a^{(2)}_{j m}(u)
     }{
     2
     } \, . \nonumber
  \end{align}
%%
We see \cref{eq:tot-from-iso} recovers the \(\pi\pi\) scattering result when \(J = 0\) (c.f. eq. (14) in \cite{Albaladejo2018}).

Similarly, inverting \cref{eq:matrix} and solving for the isospin projected amplitudes we find:
%5
  \begin{align}
    \mathcal{A}^{(I)}_\lambda(s,t,u) &=
    \sum_{j = 0}^\jmax \; (2j + 1) \; d_{\lambda 0}^j(\theta_s) \; a_{j \lambda}^{(I)}(s) \nonumber \\
    &+ \sum_{m, \, \Ip} \sum_{j = 0}^\jmax (2j + 1) \; d_{\lambda m}^J(\hat{\theta_1}) \;  d_{m 0}^j(\theta_t) \; a_{j m}^{(\Ip)}(t) \; \frac{1}{2} \, C_{I\Ip} \\
    &+ \sum_{m, \, \Ip} \sum_{j = 0}^\jmax (2j + 1) \; d_{\lambda m}^J(\hat{\theta_2}) \;  d_{m 0}^j(\theta_u) \;
     a_{j m}^{(\Ip)}(u) \; (-1)^{I + \lambda + \Ip+  m} \; \frac{1}{2} \, C_{I\Ip} \; . \nonumber
  \end{align}
%%
or with all the kinematic terms factored out,
%%
\begin{align}
  \label{eq:isobar}
  \mathcal{A}^{(I)}_\lambda&(s,t,u) =
   \xi_{\lambda 0}(z_s) \; K_{\lambda 0}(s)
  \sum_{j = 0}^\jmax \; (2j + 1) \;  (k(s)q(s))^{j - |\lambda|} \;
   \hat{d}_{\lambda 0}^j(\theta_s) \; \hat{a}_{j \lambda}^{(I)}(s) \nonumber \\
  &+ \sum_{m, \, \Ip} \frac{1}{2} \, C_{I\Ip} \;  d_{\lambda m}^J(\hat{\theta_1}) \times
  \bigg [ \xi_{m 0}(z_t) \; K_{m 0}(t) \sum_{j = 0}^\jmax (2j + 1) \;
  (k(t)q(t))^{j - |m|} \;\hat{d}_{m 0}^j(\theta_t) \; \hat{a}_{j m}^{(\Ip)}(t) \bigg]  \\
  &+ \sum_{m, \, \Ip} (-1)^{I + \lambda + \Ip + m} \; \frac{1}{2} \, C_{I\Ip} \;  d_{\lambda m}^J(\hat{\theta_2}) \times
  \bigg [\xi_{m 0}(z_u) \; K_{m 0}(u)  \sum_{j = 0}^\jmax (2j + 1) \;
  (k(u)q(u))^{j - |m|} \; \hat{d}_{m 0}^j(\theta_u) \; \hat{a}_{j m}^{(\Ip)}(u) \bigg] \;. \nonumber
\end{align}
%%
This is the final form of the isobar model for \(J^{PC} \to 3\pi\). We note from \cref{eq:isobar}, that by adding isobars into each channel separately we introduce kinematic singularities in all three channels.

\subsection{Comparison with Covariant Formalism}
%
In the covariant formalism, we write:
%%
  \begin{align}
    \label{eq:covariant-iso-decomp}
    \mathcal{A}_\lambda(s,t,u) = \epsilon_\mu^\lambda(p_A) \; \times \; &\bigg[
     \bar{F}^{(s)}(s,t,u) \; (p_1 + p_2)^\mu
     +  \bar{F}^{(t)}(s,t,u)  \; (-\bar{p}_3 + p_2)^\mu
      +  \bar{F}^{(u)}(s,t,u)  \; (p_1 - \bar{p}_3)^\mu \nonumber \\
    &+ \bar{G}^{(s)}(s,t,u) \; (p_1 - p_2)^\mu
    + \bar{G}^{(t)}(s,t,u)  \; (-\bar{p}_3 - p_2)^\mu
    + \bar{G}^{(u)}(s,t,u)  \; (p_1 + \bar{p}_3)^\mu \bigg]
  \end{align}
%%
where each \(\bar{F}^{(x)}\) and \(\bar{G}^{(x)}\) have a form analogous to \cref{eq:matching_F,eq:matching_G} respectively but with a truncated sum of isobar partial waves in the specific channel:
%%
\begin{subequations}
  \begin{align}
  \bar{F}^{(x)}(s,t,u)  &= 2 M \, \hat{a}^{(x)}_{00}(x) + \, \bigg[ \frac{2M}{\lambda_M(x)}\bigg] \times \sum_{j \text{ even}}^\infty (2j+1) \, (k(x)\,q(x))^{j} \;  \hat{d}^j_{00}(\theta_x) \; \hat{a}^{(x)}_{j0}(x)  \nonumber \\
     %%
  & \qquad \qquad \bigg[ 4 \sqrt{2} \; (x + M^2 - m_\pi^2) \;  (k(x)q(x)) \,  z_x\bigg] \times \sum_{j \text{ even}}^\infty (2j+1) \, (k(x)\,q(x))^{j - 1} \;  \hat{d}^j_{10}(\theta_x) \; \hat{a}^{(x)}_{j+}(x)
      \label{isobar-f} \\
     \bar{G}^{(x)}(s,t,u)  &=  \bigg[ - 8 \sqrt{2} \; x \; k(x) \, q(x) \bigg] \times \sum_{j = 1}^\infty (2j+1) \, (k(x)q(x))^{j - 1} \,\hat{d}^j_{10}(\theta_x) \, \hat{a}^{(x)}_{j+}(x) \; . \label{isobar-g}
  \end{align}
\end{subequations}
%%
We reiterate that the \(\hat{a}_{j\lambda}(x)\) functions in \cref{isobar-f,isobar-g} are technically not the same as partial wave amplitudes of \cref{eq:helicity}, but instead describe an isobar intermediate state with total angular momentum \(j\).

To further motivate the use of the helicity formalism of \cref{sec:helicity}, we wish to match the kinematic singularity terms in the covariant and helicity formalisms. We focus on the \(t\)-channel isobar piece of \cref{eq:isobar}, with the \(u\)-channel being analogous. Identifying the \(t\)-channel contribution of \(\mathcal{A}_\lambda\) in general:
%%
  \begin{align} \label{rot-kin-sym}
    &\sum_{m} d_{\lambda m}^J(\hat{\theta}_1) \; \xi_{m0}(z_t) \; K_{m0}(t) \times \bigg[\sum_{j=0}^\jmax (2j+1) \; (k(t)q(t))^{j-|\lambda|} \; \hat{d}^j_{m0}(\theta_t) \; \hat{a}^{(t)}_{jm}(t) \bigg]
    = \nonumber \\
%%
    &\sum_{|m| = 1}^J \bigg[ d_{\lambda \, +|m|}^J(\hat{\theta}_1) + P (-1)^{|m|} \; d_{\lambda \,-|m|}^J(\hat{\theta}_1) \bigg] \big(\xi_{|m|0}(z_t) \; K_{|m|0}(t)\big) \times
    \bigg[\sum_{j=0}^\jmax (2j+1) \; (k(t)q(t))^{j-1} \; \hat{d}^j_{|m|0}(\theta_t) \; \hat{a}^{(t)}_{j|m|}(t) \bigg] \\
%%
    &\qquad \qquad \qquad \qquad \qquad \qquad \qquad \qquad \quad
     + \frac{1}{2}\big(1 + P \big) \;
     d^J_{\lambda \, 0}(\hat{\theta}_1) \; K_{00}(z_t) \times
    \bigg[\sum_{j=0}^\jmax (2j+1) \; (k(t)q(t))^{j} \; \hat{d}^j_{00}(\theta_t) \; \hat{a}^{(t)}_{j0}(t) \bigg]
    \nonumber
  \end{align}
%%
where we have used the symmetry properties discussed in \cref{sec:symmetry}.

Immediately we see \cref{rot-kin-sym} is consistent with the \(\phi/\omega \to 3\pi\) analyses \cite{Danilkin:2014cra,Niecknig:2012sj} with \(J^{PC} = 1^{--}\), i.e. there is no \(\lambda = 0\) contribution and the kinetic singularities are the same as those of the \(s\)-channel. For the \(a_1\) decay, we have
%%
  \begin{align} \label{matrix-helicity}
    \begin{bmatrix}
  A^{(t)}_0(s,t,u) \\
  A^{(t)}_1(s,t,u)
    \end{bmatrix}
    =
    \begin{bmatrix}
        \cos \hat{\theta}_1 \;  \tilde{K}_{0}(t)   &  \sqrt{2} \; \sin \hat{\theta}_1 \; \tilde{K}_{1}(t) \\
       - \frac{1}{\sqrt{2}}  \; \sin \hat{\theta}_1 \;  \tilde{K}_{0}(t)  & \cos \hat{\theta}_1 \; \tilde{K}_{1}(t)
    \end{bmatrix}
    \times
    \begin{bmatrix}
       \hat{a}^{(t)}_{j0}(t) \\
        \hat{a}^{(t)}_{j+}(t)
    \end{bmatrix}
  \end{align}
%%
where we have condensed the notation with \(\tilde{K}_\lambda(t) = \xi_{\lambda0}(z_t) \; K_{\lambda0}(t)\) and the presence of the sum over \(j\), \(d\)-functions, and \((kq)^{j-|\lambda|}\) factors is implied. In terms of the invariant amplitudes with the momenta and helicity defined in the \(s\)-channel center of mass frame s in \cref{sec:covariant}:
%%
  \begin{align} \label{matrix-covariant}
    \begin{bmatrix}
    A^{(t)}_0(s,t,u) \\
  A^{(t)}_1(s,t,u)
    \end{bmatrix}
    =
    \begin{bmatrix}
      \epsilon^0 \cdot (- \bar{p}_3 + p_2) & \epsilon^0 \cdot (- \bar{p}_3  - p_2) \\
    \epsilon^+ \cdot (- \bar{p}_3 + p_2) & \epsilon^+ \cdot (- \bar{p}_3 - p_2)
    \end{bmatrix}
    \times
    \begin{bmatrix}
      \bar{F}^{(t)}(s,t,u)  \\
      \bar{G}^{(t)}(s,t,u)
    \end{bmatrix}
  \end{align}
%%
thus combining \cref{matrix-helicity,matrix-covariant} we get the consistency relation:
\begin{align}
  \label{consistency}
  \begin{bmatrix}
 \hat{a}^{(t)}_{j0}(t) \\
 \hat{a}^{(t)}_{j+}(t)
  \end{bmatrix}
  =
  \frac{1}{\tilde{K}_{0}(t) \; \tilde{K}_{1}(t)}
  \begin{bmatrix}
      \cos \hat{\theta}_1 \;  \tilde{K}_{1}(t)   &  - \sqrt{2} \; \sin \hat{\theta}_1 \; \tilde{K}_{1}(t) \\
     \frac{1}{\sqrt{2}}  \; \sin \hat{\theta}_1 \;  \tilde{K}_{0}(t)  & \cos \hat{\theta}_1 \; \tilde{K}_{0}(t)
  \end{bmatrix}
  \begin{bmatrix}
    \epsilon^0 \cdot (p_3 + p_2) & \epsilon^0 \cdot (p_3 - p_2) \\
  \epsilon^+ \cdot (p_3 + p_2) & \epsilon^+ \cdot (p_3 - p_2)
  \end{bmatrix}
  \begin{bmatrix}
     \bar{F}^{(t)}(s,t,u)  \\
     \bar{G}^{(t)}(s,t,u)
  \end{bmatrix}
   \; .
 \end{align}
%%
We can immediately recognize that the product of matrices as boosting the polarization Lorentz vector from the \(s\)-channel scattering frame to the \(t\)-channel frame.

Additionally, using the kinematic changes from the interchange of Mandelstam variables \cref{frame-change},
we see \cref{consistency} with \(s\leftrightarrow t\) shows
%%
\begin{align}
  \begin{bmatrix}
   \hat{a}^{(t)}_{j0}(s) \\
   \hat{a}^{(t)}_{j+}(s)
  \end{bmatrix}
  =
  \begin{bmatrix}
  \big( \tilde{K}_{0}(s) \big)^{-1} & 0 \\
  0 &   \big( \tilde{K}_{1}(s) \big)^{-1}
  \end{bmatrix}
  \begin{bmatrix}
    \epsilon^0 \cdot (p_1 + p_2) & \epsilon^0 \cdot (p_1 - p_2) \\
  \epsilon^+ \cdot (p_1 + p_2) & \epsilon^+ \cdot (p_1 - p_2)
  \end{bmatrix}
  \times
  \begin{bmatrix}
     \bar{F}^{(t)}(t,s,u)  \\
     \bar{G}^{(t)}(t,s,u)
  \end{bmatrix}
   \; .
 \end{align}
%%

%#########################################################################################################
%#########################################################################################################
%#########################################################################################################
\section{Khuri-Treiman Equations} \label{sec:unitarity}

Now we wish to impose impose elastic unitarity on each channel, via the KT equations.
Unitarity imposes a condition on the analytic structure in the complex \(s\)-plane. We assume that the isobar amplitudes only have a right-hand cut from \(s_\text{th} = 4m_\pi^2\) to \(\infty\) associated by the threshold opening of the \(\pi\pi\) final state. Unitarity tells us the discontinuity across this cut is
%%
  \begin{equation}
      \Disc \mathcal{A}^{(I)}_\lambda(s,z_s) = \frac{1}{2i} \bigg[ \mathcal{A}^{(I)}_\lambda(s + i\epsilon, z_s) - \mathcal{A}^{(I)}_\lambda(s-i\epsilon,z_s) \bigg],
  \end{equation}
%%
which we compute by assuming the reaction proceeds through a two pion intermediate state and integrating over the allowed two-body phase space, i.e.
%%
  \begin{equation}
    A(p_A) \pi(\overline{p}_3) \rightarrow \pi(q_1)\pi(q_2) \to \pi(p_1) \pi(p_2).
  \end{equation}
%%
Note the condition is on the discontinuity and not simply the imaginary part as the discontinuity is not purely imaginary in \(1 \to 3\) processes.

The starting point for out KT equations is the unitarity relation for the helicity amplitude as in \cite{Danilkin:2014cra}
%%
  \begin{align}
    \label{eq:unitarity}
    \Disc \mathcal{A}^{(I)}_\lambda(p_A \overline{p}_3 \to p_1 p_2 ) =&\; \frac{\rho(s)}{64 \pi^2} \int d\Omega_s^\prime  \; {\mathcal{T}}^{(I)^*}(q_1q_2 \to p_1p_2) \times \mathcal{A}^{(I)}_\lambda(p_A \overline{p}_{3} \to q_1 q_2 ) \nonumber \\
%%
    =& \; \frac{\rho(s)}{64 \pi^2} \int d\Omega_s^\prime  \; {\mathcal{T}}^{(I)^*}(s,z_s^{\prime\prime}) \times \mathcal{A}^{(I)}_\lambda(s,z_s^{\prime})
  \end{align}
%%
where \({\tau^{(I)}}\) is the elastic \(\pi\pi\) scatting amplitude with definite isospin-\(I\), \(\rho(s) = \sqrt{1 - 4m_\pi^2/s}\) is the two body intermediate phase space, and the integration is over the angles \(\theta^\prime\) and \(\varphi^\prime\) of the intermediate state momenta. This intermediate frame is related to the initial \(A\pi\) scattering frame (determined by \(\theta\) and \(\varphi = 0\)) by an angle, \(\theta^{\prime\prime}\), given by (see eq. 6.71 in \cite{MS})
%%
  \begin{equation}
    \cos \theta^{\prime\prime} = \cos \theta \cos \theta^\prime + \cos \varphi^\prime \sin\theta \sin \theta^\prime.
  \end{equation}
%%

For the elastic pion scattering amplitude we use the standard partial wave decomposition (see eq. 16 in \cite{Danilkin:2014cra}), using \( z_s^{\prime\prime} = \cos \theta_s^{\prime\prime}\),
%%
  \begin{align}
    \label{eq:elastic-pion}
    \mathcal{T}^{(I)}(s, z_s^{\prime\prime}) =& \; 32 \, \pi \sum_{\ell=0  }^\infty \; (2\ell+1) \, P_{\ell}(z_s^{\prime\prime}) \; \tau_\ell^{(I)}(s) \nonumber \\
%%
    =& \; 128 \, \pi^2 \; \sum_{\ell=0}^\infty \sum_{m=-\ell}^{\ell} (2\ell +1 ) Y^m_\ell(\theta_s,0) \; {Y^m_\ell}^*(\theta_s^\prime, \varphi^\prime) \; \tau_\ell^{(I)}(s) \; .
  \end{align}
%%

First we evaluate the left-hand side of \cref{eq:unitarity}. Because we assume our kinematic-singularity-free isobar functions, \(\hat{a}^{(I)}_{j \lambda}(s)\), only have a right-hand cut associated with unitarity the discontinuity comes only from the \(s\)-channel isobar:
%%
  \begin{equation}
    \label{eq:discontinuity}
    \Disc \mathcal{A}^{(I)}_\lamp = s^{\lamp/2}\, \sum_{j^\prime=0}^\jpmax \; (2 j^\prime +1) \; (k(s)q(s))^{j^\prime-\lamp}  \; \xi_{\lamp 0}(z_s)
    \; \hat{d}^{j^\prime}_{\lamp 0}(\theta_s) \; K^{j^\prime}_{\lamp0}(s) \; \Disc \hat{a}^{(I)}_{\jp \lamp}(s).
  \end{equation}
%%
Taking the \(j\)-th partial wave projection we get
%%
  \begin{equation}
    \label{eq:pw-disc}
    \frac{1}{2} \int_{-1}^1 dz_s \; P^\lambda_{j}(z_s) \; \Disc \mathcal{A}^{(I)}_\lamp =
    \bigg[ s^{\lambda/2} \, K^j_{\lambda 0}(s) \, (k(s)q(s))^{j-\lambda} \; \frac{(j+\lambda)!}{(j-
    \lambda)!} \bigg] \times \Disc \hat{a}^{(I)}_{j\lambda}(s).
  \end{equation}
%%

Next we move on to the homogeneous direct channel contribution of the right-hand side of \cref{eq:unitarity}. Combining \cref{eq:isobar,eq:elastic-pion} (note we have to be careful about the kinematic singularities we removed from the \(d\)-function in \cref{eq:halfangle} and \(\varphi^\prime \not= 0\) in \cref{eq:helicity}),
 the \(s\)-channel part of the integrand in\cref{eq:unitarity} is given by:
%%
  \begin{align}
      \label{eq:direct-channel-angle}
     \int d\Omega_s^\prime \; {Y^m_\ell}^*(\theta_s^\prime, \varphi^\prime) \times \xi_{\lambda0}(z_s^\prime) \, \hat{d}_{\lambda0}^{j^\prime}(\theta_s^\prime) \, e^{i\lambda \varphi^\prime} =&
     \; \sqrt{\frac{4\pi}{2j^\prime+1}\frac{(j^\prime+\lambda)!}{(j^\prime-\lambda)!}} \int d\Omega_s^\prime \; {Y^m_\ell}^*(\theta_s^\prime, \varphi^\prime) \;  Y^\lambda_{j^\prime}(\theta_s^\prime, \varphi^\prime) \nonumber \\
%%
    =& \;  \sqrt{\frac{4\pi}{2j^\prime+1}\frac{(j^\prime+\lambda)!}{(j^\prime-\lambda)!}} \;  \delta_{m\lambda} \; \delta_{\ell j^\prime}.
  \end{align}
%%
The entire direct-channel contribution then is
%%
  \begin{align}
    2 \; \sum_{j^\prime=0}^\jpmax \sqrt{\frac{4\pi}{2j^\prime+1}\frac{(j^\prime+\lambda)!}{(j^\prime-\lambda)!}}& \, {Y^\lambda_{j^\prime}}(\theta_s,0) \; s^{\lambda/2} \,  K^{j^\prime}_{\lambda0}(s)
    \; (2j^\prime +1) \; (k(s)q(s))^{j^\prime - \lambda}
    \bigg[\rho(s) \; \tau^{(I)^*}_j(s) \; \hat{a}^{(I)}_{j\lambda}(s) \bigg ] \nonumber \\
    =& \;  2\;  \sum_{j^\prime=0}^\jpmax \; P^\lambda_j(z_s) \; s^{\lambda/2} \, K^{j^\prime}_{\lambda0}(s)  \; (2j^\prime +1) \; (k(s)q(s))^{j^\prime - \lambda}
    \bigg[\rho(s) \; {\tau}^{(I)^*}_{j^\prime}(s) \; \hat{a}^{(I)}_{\jp\lambda}(s) \bigg ],
  \end{align}
%%
and again taking the \(j\)-th partial wave as in eq.~\ref{eq:pw-disc}, we get the direct-channel contribution:
%%
  \begin{equation}
    \label{eq:pw-direct}
   \bigg[ s^{\lambda/2} \, K^j_{\lambda 0}(s) \; (k(s)q(s))^{j - \lambda} \; \frac{(j+\lambda)!}{(j- \lambda)!} \bigg] \times \rho(s) \; {\tau}^{(I)^*}_j(s) \; \hat{a}^{(I)}_{j\lambda}(s).
  \end{equation}
%%

Finally, in the cross-channels we have the inhomogeneous part of the dispersion relation. The angular integral cannot be done analytically:
%%
  \begin{align}
   \sum_{\emp \;\Ip} \sum_{\jp=0}^\jpmax (2\jp + 1) \; C_{II^\prime} \int d\Omega^\prime_s
    \; d^J_{\lambda \emp}(\hat{\theta}^\prime_1) \; &{Y^m_\ell}^*(\theta_s^\prime, \varphi^\prime)  \; e^{i\lambda \varphi^\prime}
     \nonumber \\
    &\times \bigg[
    {t^\prime}^{\emp/2} \; \xi_{\emp 0}(z_t^\prime) \; (k(t^\prime)q(t^\prime))^{j^\prime-\emp}
    K^{j^\prime}_{\emp0}(t^\prime) \; \hat{d}_{\emp 0}^{j^\prime}(z_t^\prime) \; \hat{a}^{(\Ip)}_{j^\prime\emp}(t^\prime)
    \bigg]
  \end{align}
%%
where \(t^\prime = t^\prime(s,z_s^\prime)\) and \(z_t^\prime = z_t^\prime(s,z_s^\prime)\).
Integrating over \(\varphi^\prime\),
%%
  \begin{align}
    \label{eq:cross-1}
     \sum_{\emp \;\Ip} \sum_{\jp=0}^\jpmax (2\jp + 1) \; C_{II^\prime} \; &\sqrt{\frac{2\ell+1}{4\pi}\frac{(\ell-\emp)!}{(\ell+\emp)!}} \; \int dz_s^\prime \; P_\ell^m(z_s^\prime)
     \; d_{\lambda \emp}^J(\hat{\theta}^\prime_1) \nonumber \\
     &\times \bigg[
     {t^\prime}^{\emp/2} \; \xi_{\emp 0}(z_t^\prime) \; (k(t^\prime)q(t^\prime))^{j^\prime-\emp}
     K^{j^\prime}_{\emp0}(t^\prime) \; \hat{d}_{\emp 0}^{j^\prime}(z_t^\prime) \; \hat{a}^{(\Ip)}_{j^\prime\emp}(t^\prime)
     \bigg] \; ,
  \end{align}
%%
we can take the \(j\)-th partial wave right away. Considering only the prefactors in front of the angular integral of \cref{eq:cross-1} in \cref{eq:unitarity} coming from the second spherical harmonic in \cref{eq:elastic-pion}:
%%
  \begin{align}
    \label{eq:cross-2}
     (2\ell+1)  \; \sum_{\ell, m} \bigg [ \int \frac{dz_s}{2} &\; P^\lambda_j(z_s) \; Y^m_\ell(\theta_s,0) \bigg] \; \times \; \rho(s) \; t_\ell^*(s)
    %%
  = \; \frac{(2j+1)}{2} \; \sqrt{\frac{4\pi}{2j+1} \frac{(j+\lambda)!}{(j-\lambda)!}} \;\rho(s) \; \tau_j^{(I)*}(s) \; \delta_{\lambda m} \; \delta_{j \ell}.
  \end{align}
%%
Combining \cref{eq:cross-1,eq:cross-2}, we get the inhomogeneous contribution:
%%
  \begin{align}
      \label{eq:pw-cross}
      (2j+1) \; \rho(s) \; &\tau^{(I)*}_j(s) \; \bigg[
      \sum_{\jp \emp \Ip} (2\jp+1) \; C_{I\Ip} \;  \nonumber \\
      &\times \int d z_s^\prime \; P_j^\lambda(z_s^\prime) \times
      d_{\lambda \emp}^J(\hat{\theta}^\prime_1) \;
      {t^\prime}^{\emp/2} \; \xi_{\emp 0}(z_t^\prime) \; (k(t^\prime)q(t^\prime))^{j^\prime-\emp}
      K^{j^\prime}_{\emp0}(t^\prime) \; \hat{d}_{\emp 0}^{j^\prime}(z_t^\prime) \; \hat{a}^{(\Ip)}_{j^\prime\emp}(t^\prime)
      \bigg]
  \end{align}
%%
which we can rewrite
in terms of derivatives of Legendre polynomials and the Lorentz-covariant Kibble function, \(\phi\), by defining (with \(\lambda \geq 0\)):
%%
    \begin{equation}
    \tilde{P}_{j}^\lambda(s) = (k(s)q(s))^{j-\lambda} \, \frac{d^\lambda}{{dz_s}^\lambda} (P_j(z_s))
    \mand
    \phi = (2 \sqrt{s} \; \sqrt{1-z_s^2} \;k(s)q(s))^2 \; .
  \end{equation}
%%
With these and removing the prime on the angle, we can rewrite a form analogous to \cite{Danilkin:2014cra}
%%
  \begin{equation}
    (2j+1) \; \rho(s) \; \tau^{(I)*}_j(s) \; \bigg[
    \sum_{\jp \emp \Ip} (2\jp+1) \; C_{I\Ip} \; \int dz_s  \;
    d_{\lambda \emp}^J(\hat{\theta}^\prime_1) \times
    \bigg(  \frac{\phi}{4 \, s}\bigg)^\emp \;
    \frac{\tilde{P}^\lambda_j(s) \, \tilde{P}_{j^\prime}^\emp(t)}{(k(s)q(s))^{2j}} \;  \; \hat{a}^{(\Ip)}_{j^\prime \emp}(t)
    \bigg]
  \end{equation}
%%
Thus combining \cref{eq:pw-disc,eq:pw-direct,eq:pw-cross} we arrive at the KT equations for the kinematic singularity-free helicity amplitudes:
%%
  \begin{align}
    \Disc \hat{a}^j_\lambda(s) = \rho(s) \; t^*_{j}(s) \; & \bigg[ \; \hat{a}^j_\lambda(s) \;+ \;  2\, (2j+1) \, \frac{(j-\lambda)!}{(j+\lambda)!} \nonumber \\
    & \times \; \sum_{j^\prime = 0}^\jpmax \, (2j^\prime+1)
    \int \frac{dz_s}{2} \; \frac{{t}^{\lambda/2}
    \; K^{j^\prime}_{\lambda0}(t)}{s^{\lambda/2} \; K^j_{\lambda0}(s)}
    \frac{P^\lambda_j(z_s) \; \xi_{\lambda 0}(z_t)}{(k(s)q(s))^{j-\lambda}}
    \times (k(t)q(t))^{j^\prime-\lambda} \; \hat{d}^{j^\prime}_{\lambda0}(z_t) \; \hat{a}^{j^\prime}_{\lambda}(t) \bigg]
  \end{align}
%%
where we recall here, \(\xi_{\lambda 0}(z)\) and \(\hat{d}_{\lambda 0}^j(z)\) are given by eq.~\ref{eq:halfangle}.

We can rewrite eq.~\ref{eq:helicity-kt}
Then we write the KT equation for the \(j\)-th partial wave of a particle with helicity \(\lambda\) as:
%%
\begin{align}
  \label{eq:helicity-kt}
  \Disc \hat{a}^j_\lambda(s) = \rho(s) \; t^*_{j}(s) \, \bigg[ \; \hat{a}^j_\lambda(s) + (2j+1) \, \frac{(j-\lambda)!}{(j+\lambda)!} \sum_{j^\prime = 0}^\jpmax \, (2j^\prime+1)
  \int dz_s\;
  \bigg(  \frac{\phi}{4 \, s}\bigg)^\lambda \frac{\tilde{P}^\lambda_j(s) \, \tilde{P}_{j^\prime}^\lambda(t)}{(k(s)q(s))^{2j}} \;  \; \hat{a}^{j^\prime}_\lambda(t)\bigg]
\end{align}
%%
where we've defined the ratio of kinetic factors
%%
  \begin{equation}
    R^{jj^\prime}_{\lambda0}(s,z_s) = \frac{{t}^{\lambda/2}
    \; K^{j^\prime}_{\lambda0}(t) \; \xi_{\lambda0}(z_t)}{s^{\lambda/2} \; K^j_{\lambda0}(s) \;\xi_{\lambda0}(z_s)},
  \end{equation}
%%
with \(t = t(s,z_s)\) and \(z_t = z_t(s,z_s)\). We note that if the decaying particles has natural parity, i.e. \(P = (-1)^J\) such as the \(\phi/\omega\), then the kinematic factors are directly related to the Kibble function and are invariant between frames. Then we have \(R^{jj}_{\lambda0}(s,z_s) = 1\) and we recover eq. 18 in \cite{Danilkin:2014cra} for \(j=1\). We must also take care to remember that the kinematic functions \(K^j_{\lambda0}(x)\) may have different dependences near threshold for \(j < J\).
%#########################################################################################################
%#########################################################################################################
%#########################################################################################################
\newpage
\section{On Kinematic Constraint/Singularity Problem}
Using Collins' conventions (I think the same as Martin and Spearman) we have the parity relation
%%
  \begin{equation}
    \mathcal{A}_{-\lambda}(s,t,u) = \eta \; \mathcal{A}_{\lambda}(s,t,u)
  \end{equation}
%%
where for \(J^{PC} \to 3 \pi\), we have \(\eta = - P(-1)^{J - \lambda}\). This plays into the factorization of kinematic singularities
%%
  \begin{equation}
    \mathcal{A}_{\lambda}(s,t,u) = \tilde{K}_\lambda(s) \sum_{j = |\lambda|} (k(s)q(s))^{j-|\lambda|} \; \hat{d}_{\lambda0}^j(\theta_s) \; \hat{A}_{j\lambda}(s)
  \end{equation}
%%
because the threshold factor \(\tilde{K}_\lambda(s)\) is given by:
%%
  \begin{equation}
    \tilde{K}_\lambda(s) = s^{-|\lambda|/2} \; \xi_{\lambda0}(\theta_s) \;
      \big(\lambda^{1/2}_M(s)\big)^{|\lambda| + Y_i} \; \big(\lambda_\pi^{1/2}(s)\big)^{|\lambda| + Y_f}
  \end{equation}
%%
where
%%
  \begin{equation}
    Y_{ij}^{\pm} = \frac{1}{2} \big[
    1 - P_{i}P_{j}\eta(-1)^{\sigma_1 \pm \sigma_2} \big] -\sigma_i - \sigma_j
  \end{equation}
  means
  \begin{equation}
      Y_i = Y_{a_1 \pi} = \frac{1}{2}\big[1- P(-1)^{\lambda}\big] - J \mand Y_f = Y_{\pi\pi} = \frac{1}{2} \big[ 1 + P (-1)^{J-\lambda}] \; .
  \end{equation}
%%
\textit{This contradicts Misha's paper.} We get the \(a_1\):
%%
  \begin{align}
    \tilde{K}_{\lambda}(s) =
    \begin{cases}
       \lambda^{1/2}_M(s)    &     \text{  for }\lambda = 0, \; j=0 \\
       \lambda^{-1/2}_M(s)  &\text{ for } \lambda = 0, \; j\geq 1 \\
       s^{-1/2} \; \sin\theta_s \; \lambda_\pi(s) \; \lambda_M^{1/2}(s)    &\text{ for } \lambda = 1, \; j\geq 0
    \end{cases}
  \end{align}
%%
Writing our the scalar amplitudes and matching we get (full derivation in my complete notes on overleaf),
\begin{equation}
  G(s,t,u) =  - \sqrt{2} \; \lambda^{1/2}_\pi(s) \; \lambda^{1/2}_M(s) \;  \times \; \sum_{j = 0}^\infty (2j+1) \, (k(s)q(s))^{j - 1} \,\hat{d}^j_{10}(\theta_s) \, \hat{A}_{j+}(s)
\end{equation}
and
\begin{align}
  F(s,t,u) = 2 M \, \hat{A}_{00}(s)
  &+ \, \frac{2 \; M}{\lambda_M(s)}  \sum_{j \not= 0}^\infty (2j+1) \, (k(s)\,q(s))^{j} \;
  \times \nonumber \\
      &\bigg[\hat{d}^j_{00}(\theta_s) \; \hat{A}_{j0}(s)
   %%
+  2 \sqrt{2} \; \frac{(s + M^2 - m_\pi^2)}{M} \; \sqrt{\lambda_M(s) \; \lambda_\pi(s)} \;  z_s \; \hat{d}^j_{10}(\theta_s) \; \hat{A}_{j+}(s) \bigg] \; .
\end{align}
We note the piece in the brackets above is regular since \(\lambda(s)\) is a polynomial and has no singularities or problems at threshold. Thus the kinematic constrain is only on \(\hat{A}_{j0}(s)\) which needs to vanish at \(s = (M\pm m)^2\).
In other words the kinematic constraint is simply:
  %%
  \begin{equation}
    \hat{A}_{j0}(s) \propto \lambda_M(s) \stackrel{ s \to (M \pm m)^2 } \longrightarrow 0
  \end{equation}
  %%

\textit{This is consistent with Miguel's derivation.} Writing the crossing matrix with for helicity amplitudes for \(m\geq 0\):
%%
  \begin{align}
     \mathcal{A}_\lambda^{(s)}(s, t, u) =
       \sum_{m = 0}^J \bigg[ d_{ m \lambda}^J(\hat{\theta}_1) - P (-1)^{J+\lambda} \; d_{-m, \, \lambda }^J(\hat{\theta}_1) \bigg] \; \mathcal{A}_{m}^{(t)}(s,t,u)
  \end{align}
%%
Miguel arrives at the relations:
\begin{subequations}\begin{align}
\sqrt{2} n(s,t) F_+^{(t)}(s,t,u) + M F_0^{(t)}(s,t,u) & \stackrel{ t \to (M \pm m)^2 } \longrightarrow 0\\
\sqrt{2} n(s,t) F_+^{(s)}(s,t,u) - M F_0^{(s)}(s,t,u) & \stackrel{ s \to (M \pm m)^2 } \longrightarrow 0
\end{align}\end{subequations}
which seem paradoxical but if \(F_+(s,t,u)\) vanishes already automatically as above, then both equations are the same with the constraint:
%%
  \begin{equation}
    F_0^{(x)}(s,t,u) \propto \lambda_M(x) \stackrel{ x \to (M \pm m)^2 } \longrightarrow 0
  \end{equation}
%%
\newpage
%#########################################################################################################
%#########################################################################################################
%#########################################################################################################
\bibliography{KT-a1.bib}
%#########################################################################################################
%########################################################################################################
%########################################################################################################
\end{document}
