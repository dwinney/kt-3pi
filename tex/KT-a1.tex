\documentclass[10pt, aps,prd,amsmath,amssymb,superscriptaddress,onecolumn,
nofootinbib,showpacs,preprintnumbers]{revtex4-1}

\bibliographystyle{apsrev4-1.bst}
\usepackage{hyperref,color,subfigure}
\usepackage{amsmath,amssymb,physics,mathtools}
\usepackage[utf8]{inputenc}
\usepackage{lmodern,bm}
\usepackage{graphicx,dcolumn,tikz,xcolor,xfrac}
\usepackage{cleveref}

\newcommand{\mand}{\qquad \text{ and } \qquad}
\newcommand{\Disc}{\text{Disc }}

\begin{document}
%#########################################################################################################
%#########################################################################################################
%#########################################################################################################
\begin{center}
\large \textbf{Axial-Vector Meson, \, \(I^G J^{PC} = 1^-1^{++}\)}
\end{center}

Here we look at the axial vector decay into three pions \(A \to 3\pi\) in the Khuri-Treiman formalism. Greek indices indicate spacetime components and Latin indices represent isospin projections.
In the decay physical region of an axial-vector meson :
%%
  \begin{equation}
    \label{eq:decay-channel}
    A^d(p_A, \lambda) \rightarrow \pi_1^a(p_1) \pi_2^b(p_2) \pi^c_3(p_3)^2.
  \end{equation}
We will define invariant Mandelstam variables
%%
  \begin{align}
    s = (p_A - p_3)^2 = (p_1 + p_2)^2 \qquad \text{ and } \qquad t = (p_A - p_1)^2 = (p_3 + p).
  \end{align}
%%
By crossing symmetry, we can relate eq.~\ref{eq:decay-channel} to the \(2\to2\) scattering channel by analytic continuation. The scattering amplitude will in general depend on the isospin projection of all four particles and the helicity of the axial vector meson.
%#########################################################################################################
%#########################################################################################################
%#########################################################################################################
\section{Isospin Decomposition}
Because all the particles are isovectors, the isospin decomposition is identical to that of \(\pi\pi\to\pi\pi\), as in \cite{Albaladejo2018}:
%%
  \begin{equation}
    \label{eq:scalar-decomp}
    \mel{\pi^a(p_1)\pi^b(p_2)}{\hat{T_\lambda}}{\pi^c(\overline{p}_3)A^d(p_A)} = \delta_{ab}\delta_{cd} \, \mathcal{A}_\lambda(s,t,u) + \delta_{ac}\delta_{bd} \, \mathcal{A}_\lambda(t,s,u) + \delta_{ad}\delta_{bc} \, \mathcal{A}_\lambda(u,t,s)
  \end{equation}
%%
where the scalar amplitude \(A_\lambda(s,t,u)\) is the scalar helicity amplitude that depends only on kinematic variables and the helicity of the vector meson.

We can build isospin-definite scattering amplitudes by linear combinations of the \(s,t \text{ or } u\) channel such that
%%
	\begin{equation}\label{eq:iso-decomp}
	  \mel{\pi^a(p_1)\pi^b(p_2)}{\hat{T_\lambda}}{\pi^c(p_3)A^d(p_A)} = P^{(0)}_{abcd} \; A^{(0)}(s,t,u) + P^{(1)}_{abcd}  \; A^{(1)}(s,t,u) +  P^{(2)}_{abcd} \; A^{(2)}(s,t,u)
	\end{equation}
%%
with
%%
	\begin{equation} \label{eq:projectors}
	P^{(0)}_{abcd} = \frac{1}{3}\delta_{ab}\delta_{cd},  \quad P^{(1)}_{abcd} = \frac{1}{2}(\delta_{ac}\delta_{bd}-\delta_{ad}\delta_{bc}),  \quad \textrm{and} \quad   P^{(2)}_{abcd} = \frac{1}{2}
	(\delta_{ac}\delta_{bd} + \delta_{ad}\delta_{bc}) - \frac{1}{3} \delta_{ab}\delta_{cd}.
	\end{equation}
%%
Combining eq.~\ref{eq:iso-decomp} and \ref{eq:projectors} and comparing to eq.~\ref{eq:scalar-decomp}, we get
%%
    \begin{align} \label{eq:matrix}
      \begin{bmatrix}
      \mathcal{A}_\lambda^{(0)}(s,t,u) \\ \mathcal{A}_\lambda^{(1)} (s,t,u) \\ \mathcal{A}_\lambda^{(2)}(s,t,u)
      \end{bmatrix}
    =
      \begin{bmatrix*}[r]
        3 & 1 & 1 \\ 	0 & 1 & -1 \\ 0 & 1 & 1
      \end{bmatrix*}
      \begin{bmatrix}
      \mathcal{A}_\lambda(s,t,u) \\ \mathcal{A}_\lambda(t,s,u) \\ \mathcal{A}_\lambda(u,t,s)
      \end{bmatrix}.
    \end{align}
%%
 Each of these amplitudes will still depend on the helicity, giving us matrix equations that become unwieldy very fast. Additionally, we wish to compare to the ``freed-isobar" partial-wave analysis results from COMPASS \cite{COMPASS-Swave,Krinner:2017vch}, which extracts partial waves with an exclusive \(\pi^-\pi^-\pi^+\) final state. Therefore analysis (i.e. unitarity in section~\ref{sec:unitarity}) of will be done on the total helicity amplitudes and definite-isospin amplitudes can be constructed via eq.~\ref{eq:matrix}.
%########################################################################################################
%#########################################################################################################
%########################################################################################################
\section{Helicity Amplitudes}
Following \cite{Mikhasenko:2017rkh} We start by writing the matrix element as a sum over helicity amplitudes (see eq. 10-5 in~\cite{perl}) by considering the scattering channel \(A(p_A, \lambda) \pi_3(\overline{p}_3) \to  \pi_1(p_1)\pi_2(p_2)\):
%%
  \begin{equation}
    \label{eq:helicity}
    \mathcal{A}_\lambda = \sum_{J= 0}^\infty \, (2J +1) \; e^{i\lambda\varphi} \, d_{\lambda0}^J(\theta_s) \; a_\lambda^J(s).
  \end{equation}
%%
The angular dependence of the decay is described by Wigner-\(d\) functions of the \(s\)-channel scattering angle, \(\theta_s\), if we choose the \(x-z\) plane as the scattering plane (i.e. \(\varphi = 0 \) ).
%%
\subsection{Kinematic Singularities}
We wish to factor out all kinematic singularities in \(s\) and \(\theta_s\) from the helicity partial wave and rotational function respectively. First we define the kinematic-free \(d\)-function, denoted with a hat, such that:
%%
  \begin{equation}
      \label{eq:halfangle}
      d^J_{\lambda 0}(\theta_s) = \hat{d}^J_{\lambda 0}(\theta_s) \; \xi_{\lambda 0}(z_s) \quad \text{ where } \quad \xi_{\lambda 0}(z_s) = \bigg( \sqrt{ 1- z_s^2} \bigg)^{|\lambda|} \quad \text{ and } \quad \hat{d}^J_{\lambda 0}(\theta_s) = \frac{d^\lambda}{d \, z_s^\lambda} (P_J(z_s))
  \end{equation}
%%
where \(\xi_{\lambda 0}\) are the ``half-angle factors" (see eq. 4.4.12 in \cite{Collins}).

We similarly factor out singularities in \(s\) by defining:
%%
  \begin{equation}
    \label{eq:kinematicfreepartialwave}
    a^J_\lambda(s) = (p(s)q(s))^{J - |\lambda|} \, K_{\lambda 0} \; \hat{a}^J_\lambda(s).
  \end{equation}
%%
Here
  \begin{equation}
    \label{eq:momenta}
    q(s) = \frac{\lambda^{1/2}(m_\pi^2, m_\pi^2, s)}{2\sqrt{s}} = \frac{\lambda_\pi^{1/2}}{2\sqrt{s}} \qquad \text{ and } \qquad k(s) = \frac{\lambda^{1/2}(m_A^2, m_\pi^2, s)}{2\sqrt{s}} = \frac{\lambda_A^{1/2}}{2\sqrt{s}} ,
  \end{equation}
with \(\lambda(x,y,z) = x^2 + y^2 + z^2 - 2 (xy + yz + zx)\) is the K\"{a}ll\'{e}n function, are the magnitudes of  the relative momentum between outgoing pions and the incoming pion's momentum respectively. The \((kq)^{J-\lambda}\) term is included to cancel out singularities in \(s\) from the \(d\)-function at threshold and pseudo-threshold (see eq. 6.2.9 in \cite{Collins}).

The other kinematic factor, \(K_{\lambda0}\) arise  because near threshold \(\hat{a}_\lambda^J(s)\) has additional power behavior of \(k(s)\) or \(q(s)\) corresponding to the dependence on \(J\) and \(L\) between helicity amplitudes and \(LS\) amplitudes. Near threshold:
%%
  \begin{gather}
    a^J_\lambda(s) \sim k^{L_i}(s) \, q^{L_f}(s) \times (k(s)q(s))^{J- |\lambda|}
  \end{gather}
%%
where \(L_i\) and \(L_f\) are the minimum angular momentum of the initial and final states for the given helicity. We have:
%%
  \begin{align}
      L_i = 1 , L_f = 2& \qquad  \text{ for } \lambda = 1, \; j \geq 1 \nonumber \\
      L_i = -1, L_f = 0 &\qquad  \text{ for } \lambda = 0, \; j = 0 \nonumber \\
      L_i = 1, L_f = 0 &\qquad  \text{ for } \lambda = 0, \;  j  \geq 1 \nonumber
  \end{align}
%%
We also note that we must treat the \(J=0\) term in eq.~\ref{eq:kinematicfreepartialwave} differently, since it has a different lowest orbital angular momentum possible than \(J\geq 1\). The explicit form for \(K_{\lambda0}\) is given by Table 6.1 in \cite{Collins}:
%%
  \begin{align}
    \label{eq:k-factor}
    K_{00 }(s) =& \; m_A \, ( 2\sqrt{s} \, k(s))^{-1}\\ \nonumber
    K_{10}(s) =& \; 4 \sqrt{2} \,  s^{3/2} \, k(s) \, q^2(s)  \;
  \end{align}
%%
We must also incorporate the \(s^{\lambda/2}\) behavior that ensures the Regge poles factorize (see eq. 6.4.7 in \cite{Collins})

The amplitude must conserve parity, so since the axial-vector meson is even under parity (\( P_A = +1\) and \(P_\pi = -1\)), \(a_{+1}^J(s) = a_{-1}^J(s)\), thus we have two independent helicity amplitudes. We choose to consider \(\lambda = 0, +1\) as our helicity amplitudes.
We rewrite eq.~\ref{eq:helicity}
%%
  \begin{align}
    \label{eq:model-helicity}
    \mathcal{A}_0 &= \frac{1}{K_{00}(s)} \, \hat{a}^0_0(s) + K_{00}(s) \sum_{J \text{ even}}^\infty (2J+1) \, (k(s)q(s))^J \; \hat{d}_{00}^J(\theta_s) \, \hat{a}^J_0(s). \\
    \nonumber \\
    \mathcal{A}_+ &= \; \sqrt{s} \, \xi_{10}(z_s) \, K_{10}(s) \sum_{J \text{ even}}^\infty (2J+1) \ (k(s)q(s))^{J-1} \;  \hat{d}_{10}^J(\theta_s) \, \hat{a}^J_+(s)
  \end{align}
%%
The sums here have been restricted to only even waves by imposing Bose symmetry, \( z_s \to - z_s\) (note: \(\hat{d}^J_{00}(\theta_s) \propto P_J(z_s)\) and \(\hat{d}^J_{10}(\theta_s) \propto P^\prime_J(z_s)\) which transform under parity as \((-1)^J\) and \((-1)^{J+1}\) respectively).

We recall that \(\hat{d}_{\lambda 0}(\theta_s)\) and \(\hat{a}_\lambda^J(s)\) are free of any kinematic singularities and the analytic structure therefore is solely dynamical.
%#########################################################################################################
%#########################################################################################################
\subsection{Khuri-Treiman Equations} \label{sec:unitarity}
We now make model assumptions to evaluate the helicity amplitudes. First we use the isobar approximation to truncate the infinite sum in eq.~\ref{eq:model-helicity}. To recover some of the high-energy singularity structure we add ``isobar helicity amplitudes" in the cross channel. Note we still need to treat the \(J=0\) term differently.
%
  \begin{align}
    \label{eq:isobar}
    \frac{\mathcal{A}_\lambda}{ s^{\lambda/2} \, \xi_{\lambda0}(z_s) \, K_{\lambda 0}(s)} =& \; \sum_{J=0}^\infty (2J+1) \; (k(s)q(s))^{J-\lambda} \, \hat{d}_{\lambda0}^J(\theta_s) \, \hat{a}^J_{\lambda}(s) \nonumber  \\
    &+\sum_{J=0}^\infty (2J+1) \; \bigg[(k(t)q(t))^{J-\lambda} \, \hat{d}_{\lambda0}^J(\theta_t) \, \hat{a}^J_{\lambda}(t) + (k(u)q(u))^{J-\lambda} \, \hat{d}_{\lambda0}^J(\theta_u) \, \hat{a}^J_{\lambda}(u) \bigg]
  \end{align}
%
Now we wish to impose the KT formalism and impose elastic unitarity on each channel. To do this we assume a two pion intermediate state and integrate over the allowed phase space.
%%
  \begin{equation}
    A^d(p_A) \pi^c(\overline{p}_3) \rightarrow \pi^{a^\prime}(q_1)\pi^{b^\prime}(q_2) \to \pi^a(p_1) \pi^b(p_2)
  \end{equation}
%%
 Unitarity imposes a condition on the discontinuity along the \(s\)-channel cut opening at \(s_{\text{th}} = 4m_\pi^2\) to \(\infty\). Note the condition is on the discontinuity and not simply the imaginary part as the discontinuity is not purely imaginary in \(1 \to 3\) processes. This is because of the analytic continuation needed to relate the decay channel to the scattering channel.

The starting point for out KT equations is the unitarity relation for the helicity amplitude
%%
  \begin{align}
    \label{eq:unitarity}
    \Disc \mathcal{A}_\lambda(p_A \overline{p}_3 \to p_1 p_2 ) =&\; \frac{\rho(s)}{64 \pi^2} \int d\Omega_s^\prime  \; \tau^*(q_1q_2 \to p_1p_2) \times \mathcal{A}_\lambda(p_A \overline{p}_{3} \to q_1 q_2 ) \nonumber \\
%%
    =& \; \frac{\rho(s)}{64 \pi^2} \int d\Omega_s^\prime  \; \tau^*(s,z_s^{\prime\prime}) \times \mathcal{A}_\lambda(s,z_s^{\prime})
  \end{align}
%%
where \(\tau\) is the elastic \(\pi\pi\) scatting amplitude, and the integration is over the angles \(\theta^\prime\) and \(\varphi^\prime\) of the intermediate state momenta. This intermediate frame is related to the initial \(A\pi\) scattering frame (determined by \(\theta\) and \(\varphi = 0\)) by an angle, \(\theta^{\prime\prime}\), given by (see eq. 6.71 in \cite{MS})
%%
  \begin{equation}
    \cos \theta^{\prime\prime} = \cos \theta \cos \theta^\prime + \cos \varphi^\prime \sin\theta \sin \theta^\prime.
  \end{equation}
%%

For the elastic pion scattering amplitude we use the standard partial wave decomposition, using \( z_s^{\prime\prime} = \cos \theta_s^{\prime\prime}\),
%%
  \begin{align}
    \label{eq:elastic-pion}
    \tau^*(s, z_s^{\prime\prime}) =& \; 32 \, \pi \; I^{aba^\prime b^\prime}_{\pi\pi} \sum_{\ell=0  }^\infty \; (2\ell+1) \, P_{\ell}(z_s^{\prime\prime}) \; t_\ell^*(s) \nonumber \\
%%
    =& \; 128 \, \pi^2 \; I_{\pi\pi}^{aba^\prime b^\prime} \sum_{\ell=0}^\infty \sum_{m=-\ell}^{\ell} (2\ell +1 ) Y^m_\ell(\theta_s,0) \; {Y^m_\ell}^*(\theta_s^\prime, \varphi^\prime) \; t_\ell^*(s)
  \end{align}
%%
where \(I_{\pi\pi}\) is an isospin factor that depends on the the isospin projections of the intermediate and final state pions.

First we evaluate the left-hand side of eq.~\ref{eq:unitarity}. Because we assume our kinematic-singularity-free isobar functions, \(\hat{a}_\lambda^J(s)\), only have a right-hand cut associated with unitarity the discontinuity comes only from the \(s\)-channel isobar:
%%
  \begin{equation}
    \label{eq:discontinuity}
    \Disc \mathcal{A}_\lambda = s^{\lambda/2}\, K_{\lambda 0}(s) \; \sum_{J^\prime} \; (2 J^\prime +1) \; (k(s)q(s))^{J^\prime-\lambda}  \; \xi_{\lambda 0}(z_s)
    \; \hat{d}^{J^\prime}_{\lambda 0}(\theta_s) \; \Disc \hat{a}_\lambda^{J^\prime}(s).
  \end{equation}
%%
Taking the \(J\)-th partial wave projection we get
%%
  \begin{equation}
    \label{eq:pw-disc}
    \frac{1}{2} \int_{-1}^1 dz_s \; P^\lambda_{J}(z_s) \; \Disc \mathcal{A}_\lambda =
    \bigg[ s^{\lambda/2} \, K_{\lambda 0}(s) \, (k(s)q(s))^{J-\lambda} \; \frac{(J+\lambda)!}{(J-
    \lambda)!} \bigg] \times \Disc \hat{a}^J_\lambda(s).
  \end{equation}
%%

Next we move on to the homogeneous direct channel contribution of the right-hand side of eq.~\ref{eq:unitarity}. Combining eqs.~\ref{eq:isobar} and \ref{eq:elastic-pion} (note we have to be careful about the kinematic singularities we removed from the \(d\)-function in eq.~\ref{eq:halfangle} and \(\varphi^\prime_s \not= 0\) in eq.~\ref{eq:helicity}), the \(s\)-channel part of the integrand in eq.~\ref{eq:unitarity} is given by:
%%
  \begin{align}
      \label{eq:direct-channel-angle}
     \int d\Omega_s^\prime \; {Y^m_\ell}^*(\theta_s^\prime, \varphi^\prime) \times \xi_{\lambda0}(z_s^\prime) \, \hat{d}_{\lambda0}^{J^\prime}(\theta_s^\prime) \, e^{i\lambda \varphi^\prime} =&
     \; \sqrt{\frac{4\pi}{2J^\prime+1}\frac{(J^\prime+\lambda)!}{(J^\prime-\lambda)!}} \int d\Omega_s^\prime \; {Y^m_\ell}^*(\theta_s^\prime, \varphi^\prime) \;  Y^\lambda_{J^\prime}(\theta_s^\prime, \varphi^\prime) \nonumber \\
%%
    =& \;  \sqrt{\frac{4\pi}{2J^\prime+1}\frac{(J^\prime+\lambda)!}{(J^\prime-\lambda)!}} \;  \delta_{m\lambda} \; \delta_{\ell J^\prime}.
  \end{align}
%%
The entire direct-channel contribution then is
%%
  \begin{align}
    2 \; \sum_{J^\prime} \sqrt{\frac{4\pi}{2J^\prime+1}\frac{(J^\prime+\lambda)!}{(J^\prime-\lambda)!}}& \, {Y^\lambda_{J^\prime}}(\theta_s,0) \; s^{\lambda/2} \, K_{\lambda 0}(s) \; (2J^\prime +1) \; (k(s)q(s))^{J^\prime - \lambda}
    \bigg[\rho(s) \; t^*_J(s) \; \hat{a}^\lambda_J(s) \bigg ] \nonumber \\
    =& \;  2\;  \sum_J \; P^\lambda_J(z_s) \; s^{\lambda/2} \, K_{\lambda 0}(s) \; (2J^\prime +1) \; (k(s)q(s))^{J^\prime - \lambda}
    \bigg[\rho(s) \; t^*_{J^\prime}(s) \; \hat{a}^\lambda_{J^\prime}(s) \bigg ],
  \end{align}
%%
and again taking the \(J\)-th partial wave as in eq.~\ref{eq:pw-disc}, we get the direct-channel contribution:
%%
  \begin{equation}
    \label{eq:pw-direct}
   \bigg[ s^{\lambda/2} \, K_{\lambda 0}(s) \; (k(s)q(s))^{J - \lambda} \; \frac{(J+\lambda)!}{(J- \lambda)!} \bigg] \times \rho(s) \; t^*_J(s) \; \hat{a}_\lambda^J(s).
  \end{equation}
%%

Finally, in the cross-channels we have the inhomogeneous part of the dispersion relation. The angular integral cannot be done analytically:
%%
  \begin{align}
    \label{eq:cross-1}
  \sum_{J^\prime} (2J^\prime+1) \;&\int d\Omega_s^\prime \;   (k(t^\prime)q(t^\prime))^{J^\prime-\lambda} \;  {Y^m_\ell}^*(\theta_s^\prime, \varphi^\prime) \;
   \xi_{\lambda 0}(z_s^\prime) \, \hat{d}_{\lambda 0}^{J^\prime}(z_t^\prime) \, e^{i\lambda \varphi^\prime} \; \hat{a}^{J^\prime}_\lambda(t^\prime) \nonumber \\
%%
  &= \sum_{J^\prime} \; (2J^\prime +1) \sqrt{\frac{2\ell+1}{4\pi}\frac{(\ell-\lambda)!}{(\ell+\lambda)!}} \;
  \int dz_s^\prime \; P^\lambda_\ell(z_s^\prime) \, \xi_{\lambda 0}(z_s^\prime)
  \times (k(t^\prime)q(t^\prime))^{J^\prime-\lambda} \; \hat{d}^{J^\prime}_{\lambda0}(z_t^\prime) \; \hat{a}^{J^\prime}_{\lambda}(t^\prime)
  \end{align}
%%
where \(t^\prime = t^\prime(s,z_s^\prime)\) and \(z_t^\prime = z_t^\prime(s,z_s^\prime)\). So we take the \(J\)-th partial wave right away. Considering only the prefactors in front of the angular integral of eq.~\ref{eq:cross-1} in eq.~\ref{eq:unitarity}
%%
  \begin{align}
    \label{eq:cross-2}
     (2\ell+1) \; s^{\lambda/2} \; K_{\lambda 0}(s)  \; \sum_{\ell, m} \bigg [ \int dz_s &\; P^\lambda_J(z_s) \; Y^m_\ell(\theta_s,0) \bigg] \; \times \; \rho(s) \; t_\ell^*(s) \nonumber \\
    %%
  =& \; (2J+1) \; \sqrt{\frac{4\pi}{2J+1} \frac{(J+\lambda)!}{(J-\lambda)!}} \;  s^{\lambda/2} \; K_{\lambda 0}(s)  \;\rho(s) \; t_J^*(s) \; \delta_{\lambda m} \; \delta_{J \ell}.
  \end{align}
%%
Combining eqs.~\ref{eq:cross-1} and \ref{eq:cross-2}, we get the inhomogeneous contribution:
%%
  \begin{align}
    \label{eq:pw-cross}
     (2J+1) \; s^{\lambda/2} \; K_{\lambda 0}(s) \times \rho(s) \; t^*_{J}(s) \;
     \bigg[ \sum_{J^\prime} \, (2J^\prime+1)
     \int dz_s^\prime \; P^\lambda_J(z_s^\prime) \; \xi_{\lambda 0}(z_s^\prime)
     \times (k(t^\prime)q(t^\prime))^{J^\prime-\lambda} \; \hat{d}^{J^\prime}_{\lambda0}(z_t^\prime) \; \hat{a}^{J^\prime}_{\lambda}(t^\prime) \bigg]
  \end{align}
%%

Thus combining eqs.~\ref{eq:pw-disc},~\ref{eq:pw-direct}, and~\ref{eq:pw-cross} we arrive at the KT equations for the kinematic singularity-free helicity amplitudes (removing the prime on the integrated angle because its the only one):
%%
  \begin{align}
    \label{eq:helicity-kt}
    \Disc \hat{a}^J_\lambda(s) = \rho(s) \; t^*_{J} \; \bigg[ \; \hat{a}^J_\lambda(s) \;+ \; & (2J+1) \, \frac{(J-\lambda)!}{(J+\lambda)!} \nonumber \\
    & \times \sum_{J^\prime} \, (2J^\prime+1)
    \int dz_s \; \frac{P^\lambda_J(z_s) \; \xi_{\lambda 0}(z_s)}{(k(s)q(s))^{J-\lambda}}
    \times (k(t)q(t))^{J^\prime-\lambda} \; \hat{d}^{J^\prime}_{\lambda0}(z_t) \; \hat{a}^{J^\prime}_{\lambda}(t) \bigg]
  \end{align}
%%
where we recall here, \(\xi_{\lambda 0}(z)\) and \(\hat{d}_{\lambda 0}^J(z)\) are given by eq.~\ref{eq:halfangle}. We note also that since the kinematic factors \(K_{\lambda 0}(s)\) canceled out, eq.~\ref{eq:helicity-kt} is general for any \(\lambda\) and or \(J\) (we do not need to treat \(\lambda=0, J=0\) differently).
%#########################################################################################################
%#########################################################################################################
%#########################################################################################################
\section{Scalar Amplitudes / Form Factors}
We also write out the most general covariant structure, contracting the polarization tensor of the decaying mesons with two independent combinations of the momenta of the pions:
%%
  \begin{equation}
    \label{eq:covariant}
    \mathcal{A}_\lambda = \epsilon_\mu^\lambda(p_A) \, \bigg[ F(s,t,u) \; (p_1 + p_2)^\mu + G(s,t,u) \;  (p_1 - p_2)^\mu \bigg].
    \end{equation}
%%
where \(F_i\) are two independent, Lorentz-scalar amplitudes/form factors. The momenta of the pions are \(p_i\), and the axial vector meson, \(A\), has polarization vector \(\epsilon_\mu\) which depends on helicity, \(\lambda\), energy in the center of mass frame, \(E_A\), and momentum \(p_A\):
%%
  \begin{equation}
    \label{eq:polarization}
    \epsilon_\mu(p_A, \pm1) = \frac{1}{\sqrt{2}} \big( 0, \mp 1, - i, 0 \big) \qquad \text{ and } \qquad \epsilon_\mu(p_A, 0) = \frac{1}{m_A} \big( p_A, 0, 0, E_A \big).
    \end{equation}
%%
We also have
%%
  \begin{gather}
    \vec{p}_1 = q(s) \; (\sin \theta_s, 0,  -\cos \theta_s) \qquad \qquad \vec{p}_2 = q(s) \; (-\sin \theta_s, 0 , \cos \theta_s ) \\
    \vec{p}_3 = - \vec{p}_A = k(s) \; (0,0,-1). \nonumber
  \end{gather}
%%

The choices of tensor structure in eq.~\ref{eq:covariant} are to highlight the intrinsic Bose symmetry of the reaction. Because we have identical pions in the final states, the helicity amplitude should be invariant under the the interchange \(t \leftrightarrow u \) or \(p_1 \leftrightarrow p_2\).
%%
  \begin{equation}
    F(s,t,u) = F(s,u,t) \qquad \text{ and } \qquad G(s,t,u) = - G(s,u,t).
  \end{equation}
%%
In the center of mass frame, using eqs.~\ref{eq:momenta} and \ref{eq:polarization} for \(\lambda = 0 \text{ and } +1 \):
%%
  \begin{align}
    \label{eq:contract_zero}
    \epsilon_\mu(p_A,0) p_1^\mu &= \frac{E_1}{m_A} \, k(s) + \frac{E_A}{m_A} \,  z_s \, q(s)
    \qquad \qquad \epsilon_\mu(p_A,+1) p_1^\mu = - \frac{1}{\sqrt{2}} \, \sqrt{1 - z_s^2} \; q(s) \nonumber \\
%%
    \epsilon_\mu(p_A,0) p_2^\mu &= \frac{E_2}{m_A} \, k(s) - \frac{E_A}{m_A} \,  z_s \, q(s)
    \qquad \qquad \epsilon_\mu(p_A,+1) p_2^\mu =  \frac{1}{\sqrt{2}} \, \sqrt{1 - z_s^2} \; q(s)  \\
%%
    \epsilon_\mu(p_A,0) p_3^\mu &= \frac{k(s)}{m_A} \, \big[E_3 + E_A \big]
    \qquad \qquad \qquad \quad \epsilon_\mu(p_A,+1) p_3^\mu = 0 \nonumber
  \end{align}
%%
We also have
%%
  \begin{equation}
    \label{eq:energies}
    E_1 = E_2 = \frac{\sqrt{s}}{2} \mand E_A = \frac{s + m_A^2 - m_\pi^2}{2 \sqrt{s}}
  \end{equation}
Using eq.~\ref{eq:contract_zero} in eq.~\ref{eq:covariant}, we get the helicity amplitude in terms of its covariant form factors to directly compare with eq.~\ref{eq:helicity}:
%%
 \begin{align}
  \label{eq:covariant_zero}
   \mathcal{A}_0 =& \; \frac{\sqrt{s}}{m_A} \, k(s) \; F(s,t,u) + 2 \; \frac{E_A}{m_A} \, z_s \; q(s) \; G(s,t,u) \\
   \nonumber \\
   \label{eq:covariant_plus}
   \mathcal{A}_+ =& \; \sqrt{2} \sqrt{1-z_s^2} \, q(s) \; G(s,t,u).
 \end{align}
%%
Comparing eq.~\ref{eq:covariant_zero} with eq.~\ref{eq:helicity} and using eqs.~\ref{eq:halfangle} and \ref{eq:k-factor}, we can match:
%%
  \begin{equation}
    \label{eq:matching_G}
    G(s,t,u) =  \; \lambda_A^{1/2}  \, \lambda_\pi^{1/2}  \, s \sum_{J \text{ even}}^\infty (2J+1) \, (k(s)q(s))^{J-1} \,\hat{d}^J_{10}(\theta_s) \, \hat{a}^J_+(s)
  \end{equation}
%%
Matching the other helicity amplitude and using eq.~\ref{eq:matching_G},
%%
  \begin{align}
    \label{eq:matching_F}
    F(s,t,u) = \; \hat{a}^0_0 + \, \frac{1}{\lambda_A} \sum_{J \text{ even}}^\infty (2J+1) \, (k(s)\,q(s))^{J} \; \bigg[ \hat{d}^J_{00}(\theta_s) \; \hat{a}^J_0(s)
%%
  - \lambda_\pi \, \lambda_A^{1/2} \; (s + m_A^2 - m_\pi^2)\,  z_s \; \hat{d}^J_{10}(\theta_s) \; \hat{a}^J_+(s) \bigg]
  \end{align}
%%
Here we recall that \(\hat{a}^J_+\) and \(\hat{a}^J_0\) vanish for odd values of \(J\). Additionally since we want \(F\) and \(G\) to be free of any kinematic singularities, the \(\lambda_A^{-1}\) term in front of the sum for \(J\geq 1\) poses a problem as it adds two poles at \(s = (m_A^2 \pm m_\pi^2)\). This means our helicity amplitudes are not completely independent of each other as the
terms in the brackets in eq.~\ref{eq:matching_F} must vanish at these points.
%#########################################################################################################
%#########################################################################################################
%#########################################################################################################
\bibliography{KT-a1.bib}
%#########################################################################################################
%########################################################################################################
%########################################################################################################
\end{document}
