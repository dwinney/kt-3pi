%#########################################################################################################
%#########################################################################################################
%#########################################################################################################
\begin{center}
\large \textbf{Axial-Vector Meson, \, \(I^G J^{PC} = 1^-1^{++}\)}
\end{center}
Here we look at the axial vector decaying into three pions in the Khuri-Treiman formalism, however we will try to keep formulas more general in terms of the mass of the decaying particle, \(M\), and its quantum numbers, \(I^G \, J^{PC}\). Greek indices indicate spacetime components and Latin indices represent isospin projections.
In the decay physical region of an axial-vector meson :
%%
  \begin{equation}
    \label{eq:decay-channel}
    M^d(p_M, \lambda) \rightarrow \pi_1^a(p_1)\;  \pi_2^b(p_2) \; \pi^c_3(p_3),
  \end{equation}
we define the usual invariant, Mandelstam variables
%%
  \begin{align} \label{eq:invariants}
    s = (p_M - p_3)^2,  \qquad \qquad t = (p_M - p_1)^2,  \qquad  \qquad u = (p_M-p_2)^2.
  \end{align}
%%
Using crossing symmetry, we can relate \cref{eq:decay-channel} in the physical decay region to the \(2\to2\) scattering channel by analytic continuing the energy variables. The scattering amplitude will in general depend on the isospin projection of all four particles and the helicity of the axial vector meson.

Because all the particles are isovectors, the isospin decomposition is identical to that of \(\pi\pi\to\pi\pi\), as in \cite{Albaladejo2018}:
%%
  \begin{equation}
    \mel{\pi^a(p_1)\pi^b(p_2)}{\hat{T_\lambda}}{\pi^c(\overline{p}_3)A^d(p_A)} = \delta_{ab}\delta_{cd} \; \mathcal{A}^{c}_\lambda(s,t,u) + \delta_{ad}\delta_{bc} \; \mathcal{A}^{a}_\lambda(s,t,u) + \delta_{ac}\delta_{bd} \; \mathcal{A}^{b}_\lambda(s,t,u)
  \end{equation}
%%
  here however the scalar amplitudes \(A^c_\lambda(s,t,u)\) depends on kinematic variables and additionally the helicity of the vector meson and the isospin projection of the pion in the initial state.

Because the different initial state pion projections should be related by crossing, we can relate all three scalar amplitudes in \cref{eq:scalar-decomp} to a single amplitude, \(\mathcal{A}_\lambda(s,t,u) \equiv \mathcal{A}_\lambda^{c}(s,t,u)\) by permutations of Mandelstam variables,
%%
\begin{equation}
  \label{eq:scalar-decomp}
  \mel{\pi^a(p_1)\pi^b(p_2)}{\hat{T_\lambda}}{\pi^c(\overline{p}_3)M^d(p_A)} = \delta_{ab}\delta_{cd} \; \mathcal{A}_\lambda(s,t,u) + \delta_{ad}\delta_{cb} \; \mathcal{A}_\lambda(t,s,u) + \delta_{ac}\delta_{bd} \; \mathcal{A}_\lambda(u,t,s).
\end{equation}
%%
We note that we thus only have two independent scalar amplitudes corresponding to the two unique helicity projections of the axial vector (see \cref{sec:kin-singularities}).

We can build isospin-definite scattering amplitudes by linear combinations of the \(s, \; t \text{ or } u\) channel such that
%%
	\begin{equation}\label{eq:iso-decomp}
	  \mel{\pi^a(p_1)\pi^b(p_2)}{\hat{T_\lambda}}{\pi^c(p_3)M^d(p_M)} = P^{(0)}_{abcd} \; A_\lambda^{(0)}(s,t,u) + P^{(1)}_{abcd}  \; A_\lambda^{(1)}(s,t,u) +  P^{(2)}_{abcd} \; A_\lambda^{(2)}(s,t,u)
	\end{equation}
%%
with projectors,
%%
	\begin{equation} \label{eq:projectors}
	P^{(0)}_{abcd} = \frac{1}{3}\delta_{ab}\delta_{cd},  \quad P^{(1)}_{abcd} = \frac{1}{2}(\delta_{ac}\delta_{bd}-\delta_{ad}\delta_{bc}),  \quad \textrm{and} \quad   P^{(2)}_{abcd} = \frac{1}{2}
	(\delta_{ac}\delta_{bd} + \delta_{ad}\delta_{bc}) - \frac{1}{3} \delta_{ab}\delta_{cd}.
	\end{equation}
%%
Combining\cref{eq:iso-decomp,eq:projectors} and comparing to \cref{eq:scalar-decomp}, we get
%%
    \begin{align} \label{eq:matrix}
      \begin{bmatrix}
      \mathcal{A}_\lambda^{(0)}(s,t,u) \\ \mathcal{A}_\lambda^{(1)} (s,t,u) \\ \mathcal{A}_\lambda^{(2)}(s,t,u)
      \end{bmatrix}
    =
      \begin{bmatrix*}[r]
        3 & 1 & 1 \\ 	0 & 1 & -1 \\ 0 & 1 & 1
      \end{bmatrix*}
      \begin{bmatrix}
      \mathcal{A}_\lambda(s,t,u) \\ \mathcal{A}_\lambda(t,s,u) \\ \mathcal{A}_\lambda(u,t,s)
      \end{bmatrix}.
    \end{align}
%%
 Each of these amplitudes will still depend on the helicity, giving us matrix equations that become unwieldy very fast. We wish to compare to the ``freed-isobar" partial-wave analysis results from COMPASS \cite{COMPASS-Swave,Krinner:2017vch}, which extracts partial waves with an exclusive \(\pi^-\pi^-\pi^+\) final state.
 %#########################################################################################################
 %#########################################################################################################
 %#########################################################################################################
