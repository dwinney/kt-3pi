%#######################################)#################################################################
%#########################################################################################################
%#########################################################################################################
\section{Isobar Decomposition} \label{sec:isobar-decomp}
In order to evaluate the helicity amplitudes, use the isobar approximation to truncate the infinite sum in \cref{eq:model-helicity-zero,eq:model-helicity-plus}. To recover some of the high-energy singularity structure we add ``isobar helicity amplitudes" in the cross channel
%%
  \begin{equation}
    \label{eq:isobar-def}
    \mathcal{A}_\lambda(s,t,u) = A_\lambda^{(s)}(s,t,u) +  A_\lambda^{(t)}(s,t,u) + A_\lambda^{(u)}(s,t,u) \; .
  \end{equation}
%%
Here each \(A_\lambda^{(x)}\) represents a finite sum of isobar partial waves in the specified scattering channel.

 Because we are considering the decay of a particle with spin, there is additional complications of the helicity being defined in different channels. Evaluating everything in the \(s\)-channel center of mass frame, \(A_\lambda^{(s)}\) is analogous  to \cref{eq:helicity-final} with the sum truncated at some \(j_\text{max}\), however the \(t\) and \(u\) channel isobars introduce additional rotational functions.
 In general we may write (for now ignoring kinematic singularities):
 %%
 \begin{align}
   \label{eq:iso-D-matrix}
    \mathcal{A}_\lambda(s,t,u) &= \sum_{m} D^{J^*}_{\lambda m}(r_3)
    \bigg [
    \sum_{j = 0}^{\jmax} (2j+1) \; d_{\lambda0}^j(\theta_s) \; a_{ j m}^{(s)}(s)
    \bigg ]
     \nonumber \\
    &+ \sum_{m} D^{J^*}_{\lambda m}(r_1) \;
    \bigg[
    \sum_{j = 0}^{\jmax} (2j+1) \;d_{m0}^j (\theta_t) \; a_{j m}^{(t)}(t)
    \bigg]
    + \sum_{m} D^{J^*}_{\lambda m}(r_2) \;
    \bigg[
    \sum_{j = 0}^{\jmax} (-1)^{j + m} \; (2j+1) \;d_{m0}^j(\theta_u) \; a_{j m}^{(u)}(u)
    \bigg] \; .
 \end{align}
 %%
 The \(D\)-matrix arguments, \(r_i\), denote a rotation by a set of Euler angles, \((\phi_i,\theta_i,\psi_i)\), that take the \(A \, \pi_3\) CM frame (the \(s\)-channel helicity frame) to the \(A \, \pi_i\) CM frame. Since the scattering process is planar, we can in general choose \(\phi_i, \,\psi_i = 0\), with the only transformations between the different helicity frames are boosts and rotation by \(\theta_i\) to align \(p_i\) with the \(z\)-direction (opposite \(p_A\)). We note that the additional Wigner rotation in the \(s\)-channel isobars vanishes because \(\hat{\theta_3} \equiv 0\)
 in the \(s\)-channel helicity frame.

We can then rewrite \cref{eq:iso-D-matrix} as
%%
\begin{align}
  \label{eq:iso-d-func}
   \mathcal{A}_\lambda(s,t,u) &= \sum_{j = 0}^{\jmax} (2 j+1) \; d_{\lambda0}^j(\theta_s) \; a_{j \lambda}^{(s)}(s)
    \nonumber \\
   &+ \sum_{m} \sum_{j = 0}^{\jmax} (2j+1) \;
    d^{J}_{\lambda m}(\hat{\theta}_1)
    \;d_{m0}^j(\theta_t) \; a_{j m}^{(t)}(t)
    %%
   + \sum_{m} \sum_{j = 0}^{\jmax} (-1)^{j + \lambda + m} \; (2j+1) \;
    d^{J}_{\lambda m}(\hat{\theta}_2)
    \;d_{m0}^j(\theta_u) \; a_{j m}^{(u)}(u) \; ,
  \end{align}
%%
where \(\hat{\theta}_i\) the angle between \(p_3\) and \(p_i\) in the total CM frame \((p_M = 0\)):
%%
  \begin{align}
    \sin\hat{\theta}_1 = \frac{
    2 \; M \; \sqrt{\phi}
    }{
    \sqrt{\lambda_M(s) \;  \lambda_M(t)}
    }
    \mand
    \sin\hat{\theta}_2 = \frac{
    2 \; M \; \sqrt{\phi}
    }{
    \sqrt{\lambda_M(s) \;  \lambda_M(u)}
    }
  \end{align}
%%
or
%%
  \begin{align}
    \cos\hat{\theta}_1 &= \frac{n(s,t)}
    {\sqrt{\lambda_M(s) \; \lambda_M(t)}} \mand
    \cos\hat{\theta}_2 = \frac{n(s,u)}
    {\sqrt{\lambda_M(s) \; \lambda_M(u)}} \; .
  \end{align}
%%
for \(n(s,t) = (M^2 + m_\pi^2 - s)(M^2 + m_\pi^2 - t) + 2 \; M^2 (2m_\pi^2 - u)\).

 We note that the functions \(a^{(x)}_{jm}(x)\) are not the same as the helicity partial waves of the previous section. The isobar partial wave amplitudes describe an isobar intermediate state with total angular momentum \(j\), but taking the partial wave projection of \cref{eq:iso-d-func} we see we have contributions from the cross channel isobar functions as well, which we did not have in \cref{sec:helicity}.

Particles with unnatural parity which decay into \(3\pi\) correspond to isovector particles, thus we need to take into account the different isospin projections in which the reaction can resonate.
We define isospin definite isobars by plugging \cref{eq:iso-d-func} into \cref{eq:matrix} and identifying the \(s\) dependent piece:
%%
\begin{align}
  \begin{bmatrix}
  a^{(0)}_{j\lambda}(x) \\ a^{(1)}_{j\lambda}(x) \\ a^{(2)}_{j\lambda}(x)
  \end{bmatrix}
=
  \begin{bmatrix*}[r]
    3 & 1 & 1 \\ 	0 & 1 & -1 \\ 0 & 1 & 1
  \end{bmatrix*}
  \begin{bmatrix}
  a^{(s)}_{j\lambda}(x) \\ a^{(t)}_{j\lambda}(x) \\ a^{(u)}_{j\lambda}(x)
  \end{bmatrix}.
\end{align}
%%
We note this is the same definition as \cite{Albaladejo2018}, which did not have complication of spin in the \(J^P = 0^-\) case. This is because the interchanges of Mandelstam variables are related to the interchanges of particles by
%%
  \begin{equation}
    \label{frame-change}
    s\leftrightarrow  t  \qquad \pi_3 \leftrightarrow \pi_1 \qquad \Rightarrow \qquad \hat{\theta}_1 \to \hat{\theta}_3 = 0 \; ,
  \end{equation}
%%
and thus the sum over helicities for the \(t\)-channel isobars vanishes, \(d_{\lambda m}^J(0) = \delta_{\lambda m} \).

We can then relate \(\mathcal{A}_\lambda(s,t,u)\) to the isobar amplitudes of definite isospin:
%%
  \begin{align}
    \label{eq:tot-from-iso}
    \mathcal{A}_\lambda(s,t,u) &= \sum_{j = 0}^\jmax
      (2j + 1) \; d_{\lambda 0}^j(\theta_s) \;
      \frac{
     a_{j \lambda}^{(0)}(s) - a_{j \lambda}^{(2)}(s)
     }{
     3
     } \nonumber \\
     &+ \sum_m \sum_{j=0}^\jmax \; (2j+1)
     \; d^{J}_{\lambda m}(\hat{\theta}_1) \; d_{m0}^j(\theta_t) \;
     \frac{
     a^{(1)}_{j m}(t) + a^{(2)}_{j m}(t)
     }{
     2
     } \\
     &+ \sum_m \sum_{j=0}^\jmax \; (-1)^{j + m + \lambda} \; (2j+1)
     \; d^{J}_{\lambda m}(\hat{\theta}_2) \; d_{m0}^j(\theta_u) \;
     \frac{
     a^{(1)}_{jm}(u) + a^{(2)}_{j m}(u)
     }{
     2
     } \, . \nonumber
  \end{align}
%%
We see \cref{eq:tot-from-iso} recovers the \(\pi\pi\) scattering result when \(J = 0\) (c.f. eq. (14) in \cite{Albaladejo2018}).

Similarly, inverting \cref{eq:matrix} and solving for the isospin projected amplitudes we find:
%5
  \begin{align}
    \mathcal{A}^{(I)}_\lambda(s,t,u) &=
    \sum_{j = 0}^\jmax \; (2j + 1) \; d_{\lambda 0}^j(\theta_s) \; a_{j \lambda}^{(I)}(s) \nonumber \\
    &+ \sum_{m, \, \Ip} \sum_{j = 0}^\jmax (2j + 1) \; d_{\lambda m}^J(\hat{\theta_1}) \;  d_{m 0}^j(\theta_t) \; a_{j m}^{(\Ip)}(t) \; \frac{1}{2} \, C_{I\Ip} \\
    &+ \sum_{m, \, \Ip} \sum_{j = 0}^\jmax (2j + 1) \; d_{\lambda m}^J(\hat{\theta_2}) \;  d_{m 0}^j(\theta_u) \;
     a_{j m}^{(\Ip)}(u) \; (-1)^{I + \lambda + \Ip+  m} \; \frac{1}{2} \, C_{I\Ip} \; . \nonumber
  \end{align}
%%
or with all the kinematic terms factored out,
%%
\begin{align}
  \label{eq:isobar}
  \mathcal{A}^{(I)}_\lambda&(s,t,u) =
   K_{\lambda}(s)
  \sum_{j = 0}^\jmax \; (2j + 1) \;  (k(s)q(s))^{j - |\lambda|} \;
   \hat{d}_{\lambda 0}^j(\theta_s) \; \hat{a}_{j \lambda}^{(I)}(s) \nonumber \\
  &+ \sum_{m, \, \Ip} \frac{1}{2} \, C_{I\Ip} \;  d_{\lambda m}^J(\hat{\theta_1}) \times
  \bigg [  K_{m}(t) \sum_{j = 0}^\jmax (2j + 1) \;
  (k(t)q(t))^{j - |m|} \;\hat{d}_{m 0}^j(\theta_t) \; \hat{a}_{j m}^{(\Ip)}(t) \bigg]  \\
  &+ \sum_{m, \, \Ip} (-1)^{I + \lambda + \Ip + m} \; \frac{1}{2} \, C_{I\Ip} \;  d_{\lambda m}^J(\hat{\theta_2}) \times
  \bigg [ K_{m}(u) \sum_{j = 0}^\jmax (2j + 1) \;
  (k(u)q(u))^{j - |m|} \; \hat{d}_{m 0}^j(\theta_u) \; \hat{a}_{j m}^{(\Ip)}(u) \bigg] \;. \nonumber
\end{align}
%%
This is the final form of the isobar model for \(J^{PC} \to 3\pi\). We note from \cref{eq:isobar}, that by adding isobars into each channel separately we introduce kinematic singularities in all three channels.
%#########################################################################################################
%#########################################################################################################
%#########################################################################################################
