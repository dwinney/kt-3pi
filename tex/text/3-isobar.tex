%#######################################)#################################################################
%#########################################################################################################
%#########################################################################################################
\section{Isobar Decomposition} \label{sec:isobar-decomp}
In order to evaluate the helicity amplitudes, use the isobar approximation to truncate the infinite sum in \cref{eq:model-helicity-zero,eq:model-helicity-plus}. To recover some of the high-energy singularity structure we add ``isobar helicity amplitudes" in the cross channel
%%
  \begin{equation}
    \label{eq:isobar-def}
    \mathcal{A}_\lambda(s,t,u) = \mathcal{A}_\lambda^{(s)}(s,t,u) +  \mathcal{A}_\lambda^{(t)}(s,t,u) + \mathcal{A}_\lambda^{(u)}(s,t,u) \; .
  \end{equation}
%%
Here each \(\mathcal{A}_\lambda^{(x)}\) represents a finite sum of isobar partial waves in the specified scattering channel:
%%
  \begin{equation}
    \mathcal{A}_\lambda^{(x)}(s,t,u) = \sum_{j = 0}^\jmax \; (2j+1) \; d_{\lambda0}^j(\theta_x) \; a^{(x)}_{j\lambda}(x) \; .
  \end{equation}
%5
We use lowercase, \(a_{j\lambda}(x)\) to differentiate the helicity \textit{isobar} amplitudes from the helicity partial wave amplitudes of \cref{eq:helicity}.

 Because we are considering the decay of a particle with spin, there is additional complications of the helicity being defined in different channels. Evaluating everything in the \(s\)-channel center of mass frame, \(\mathcal{A}_\lambda^{(s)}\) is analogous  to \cref{eq:helicity-final} with the sum truncated at some \(j_\text{max}\), however the \(t\) and \(u\) channel isobars introduce additional rotational functions from crossing (c.f. \cref{sec:crossing}).

 Using \cref{eq:crossing-relation}, we may write (for now ignoring kinematic singularities):
%%
\begin{align}
  \label{eq:iso-d-func}
   \mathcal{A}_\lambda(s,t,u) &= \sum_{j = |\lambda|}^{\jmax} (2 j+1) \; d_{\lambda0}^j(\theta_s) \; a_{j \lambda}^{(s)}(s)
    \nonumber \\
  %%
   &+ \sum_\lamp (-1)^\lamp \; d_{\lamp \lambda}^J(\hat{\theta}_1) \times
   \bigg[
   \sum_{\jp =  |\lamp|}^{\jpmax} (2\jp+1) \; d_{\lamp0}^\jp(\theta_t) \; a_{\jp \lamp}^{(t)}(t)
   \bigg] \\
  %%
  &+  (-1)^\lambda \sum_{\lamp} (-1)^{\lamp} \; d^{J}_{\lamp\lambda }(\hat{\theta}_2)
  \times
  \bigg[
  \sum_{\jp =  |\lamp|}^{\jpmax} \, (2\jp+1) \; d_{\lamp0}^\jp(\theta_u) \;  (-1)^{\jp + \lamp}
    \, a_{\jp \lamp}^{(u)}(u)
  \bigg] \; ,
  \nonumber
  \end{align}
%%
where \(\hat{\theta}_i\) the angle between \(p_3\) and \(p_i\) in the total CM frame \((p_M = 0\)) are the same as the crossing angles \cref{sin-hat}. The relative factors between the \(t\)- and \(u\)-channel isobars are required to maintain Bose symmetry (see \cref{sec:symmetry}) with relation between cross-channel isobar \(a_{j\lambda}^{(t)}(t) = (-1)^{j + \lambda} \; a_{j\lambda}^{(u)}(t)\).

 We reemphisize that the functions \(a^{(x)}_{j\lambda}(x)\) are not the same as the helicity partial waves, \(A_{j\lambda}(x)\) of the previous section. The helicity isobar amplitudes describe an quasi-two-body intermediate state with total angular momentum \(j\). Taking the partial wave projection of \cref{eq:iso-d-func} we see we have necesarily have contributions from the cross channel isobar functions as well, which we did not have in \cref{sec:helicity}.

Particles with unnatural parity which decay into \(3\pi\) correspond to isovector particles, thus we need to take into account the different isospin projections in which the reaction can resonate (we consider the isoscalar case in the context of the \(f_2(1270)\) decay in \cref{app:2pp}).
We define isospin definite isobars by plugging \cref{eq:iso-d-func} into \cref{eq:matrix} and identifying the \(s\) dependent piece:
%%
\begin{align}
  \begin{bmatrix}
  a^{(0)}_{j\lambda}(x) \\ a^{(1)}_{j\lambda}(x) \\ a^{(2)}_{j\lambda}(x)
  \end{bmatrix}
=
  (-1)^\lambda
  \begin{bmatrix*}[r]
    3 & 1 & 1 \\ 	0 & 1 & -1 \\ 0 & 1 & 1
  \end{bmatrix*}
  \begin{bmatrix}
  a^{(s)}_{j\lambda}(x) \\ a^{(t)}_{j\lambda}(x) \\ a^{(u)}_{j\lambda}(x)
  \end{bmatrix}.
\end{align}
%%
We note this is the same definition as \cite{Albaladejo2018}, which did not have complication of spin in the \(J^P = 0^-\) case. This is because the interchanges of Mandelstam variables are related to the interchanges of particles by
%%
  \begin{equation}
    \label{frame-change}
    s\leftrightarrow  t  \qquad \pi_3 \leftrightarrow \pi_1 \qquad \Rightarrow \qquad \hat{\theta}_1 \to \hat{\theta}_3 = 0 \; ,
  \end{equation}
%%
and thus the sum over helicities for the \(t\)-channel isobars vanishes, \(d_{\lamp \lambda}^J(0) = \delta_{\lamp\lambda} \).

We can then relate \(\mathcal{A}_\lambda(s,t,u)\) to the isobar amplitudes of definite isospin:
%%
  \begin{align}
    \label{eq:tot-from-iso}
    \mathcal{A}_\lambda(s,t,u) &= \sum_{j = |\lambda|}^{\jmax} (2 j+1) \; d_{\lambda0}^j(\theta_s) \;
     \frac{a_{j \lambda}^{(0)}(s) - a_{j \lambda}^{(2)}(s)}{3}
     \nonumber \\
   %%
    &+ \sum_\lamp  d_{\lamp \lambda}^J(\hat{\theta}_1) \times
    \bigg[
    \sum_{\jp =  |\lamp|}^{\jpmax} (2\jp+1) \; d_{\lamp0}^\jp(\theta_t) \;
    \frac{a_{\jp \lamp}^{(1)}(t) + a_{\jp \lamp}^{(2)}(t)}{2}
    \bigg] \\
   %%
   &+ (-1)^\lambda \sum_{\lamp}  d^{J}_{\lamp\lambda }(\hat{\theta}_2)
   \times
   \bigg[
   \sum_{\jp =  |\lamp|}^{\jpmax} \, (2\jp+1) \; (-1)^{\jp + \lamp}
     \;d_{\lamp0}^\jp(\theta_u) \;
     \frac{a_{\jp \lamp}^{(1)}(u) + a_{\jp \lamp}^{(2)}(u)}{2}
   \bigg] \; ,
   \nonumber
   \end{align}
%%
which we see \cref{eq:tot-from-iso} recovers the \(\pi\pi\) scattering result when \(J = 0\) (c.f. eq. (14) in \cite{Albaladejo2018}).

Similarly, inverting \cref{eq:matrix} and solving for the isospin projected amplitudes we find:
%5
  \begin{align}
    \mathcal{A}^{(I)}_\lambda(s,t,u) &=
    \sum_{j = |\lambda|}^\jmax \; (2j + 1) \; d_{\lambda 0}^j(\theta_s) \; a_{j \lambda}^{(I)}(s) \nonumber \\
    %%
    &+ \sum_\lamp d_{\lamp \lambda}^J(\hat{\theta}_1) \times
    \bigg[
    \sum_\Ip \frac{1}{2} \, C_{I\Ip}
    \sum_{\jp =  |\lamp|}^\jpmax (2\jp + 1) \; d_{\lamp 0}^\jp(\theta_t) \; a_{\jp \lamp}^{(\Ip)}(t)
    \bigg] \\
    %%
    &+ (-1)^{I + \lambda} \, \sum_\lamp  d_{\lamp \lambda}^J(\hat{\theta}_2) \times
    \bigg[
     \sum_\Ip \, \frac{1}{2} \, C_{I\Ip}
    \sum_{\jp = |\lamp|}^\jpmax (2\jp + 1) \;d_{\lamp 0}^\jp(\theta_u) \;  (-1)^{\Ip + \lamp} \,  a_{\jp \lamp}^{(\Ip)}(u)
    \bigg] \nonumber \; .
  \end{align}
%%
We see that if the sum over \(j\) is restricted by setting \(a_{j\lambda}^{(I)} \equiv 0 \) unless \(j + I\) is even, then we have \(\mathcal{A}_\lambda^{(I)}(s,t,u) = (-1)^{I + \lambda} \, \mathcal{A}_\lambda^{(I)}(s,u,t)\) as required by Bose symmetry. Replacing the kinematic singularities in the form of the \(K\)-functions, \cref{eq:k-factor}, and barrier factors, we have the decomposition into kinematic-singularity-free isobar helicity amplitudes:
%%
\begin{align}
  \label{eq:isobar}
    \mathcal{A}^{(I)}_\lambda(s,&t,u) =
    K_\lambda(s) \; \sum_{j = |\lambda|}^\jmax \; (2j + 1) \; (k(s)q(s))^{j-|\lambda|} \; \hat{d}_{\lambda 0}^j(\theta_s) \; \hat{a}_{j \lambda}^{(I)}(s) \nonumber \\
    %%
    &+ \sum_\lamp  d_{\lamp \lambda}^J(\hat{\theta}_1) \times
    K_\lamp(t)
    \bigg[
    \sum_\Ip \frac{1}{2} \, C_{I\Ip} \;
    \sum_{\jp = |\lamp|}^\jpmax (2\jp + 1) \; (k(t)q(t))^{\jp-|\lamp|} \; \hat{d}_{\lamp 0}^\jp(\theta_t) \; \hat{a}_{\jp\lamp}^{(\Ip)}(t)
    \bigg] \\
    %%
    &+ (-1)^{I + \lambda} \, \sum_\lamp  d_{\lamp \lambda}^J(\hat{\theta}_2) \times
    K_\lamp(u)
    \bigg[
    \sum_\Ip  \frac{1}{2} \, C_{I\Ip} \;
    \sum_{\jp =|\lamp|}^\jpmax (2\jp + 1) \;  (k(u)q(u))^{\jp-|\lamp|} \; \hat{d}_{\lamp 0}^\jp(\theta_u)
    \;  (-1)^{\Ip + \lamp} \,\hat{a}_{\jp\lamp}^{(\Ip)}(u)
    \bigg] \nonumber \; .
  \end{align}
%%
This is the final form of the isobar model for \(J^{PC} \to 3\pi\). We note from \cref{eq:isobar}, that by adding isobars into each channel separately we introduce kinematic singularities in all three channels.
%#########################################################################################################
%#########################################################################################################
%#########################################################################################################
