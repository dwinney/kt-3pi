%#########################################################################################################
%#########################################################################################################
%#########################################################################################################
\section{Khuri-Treiman Equations} \label{sec:unitarity}

Now we wish to impose impose elastic unitarity on each channel, via the KT equations.
Unitarity imposes a condition on the analytic structure in the complex \(s\)-plane. We assume that the isobar amplitudes only have a right-hand cut from \(s_\text{th} = 4m_\pi^2\) to \(\infty\) associated by the threshold opening of the \(\pi\pi\) final state. Unitarity tells us the discontinuity across this cut is
%%
  \begin{equation}
      \Disc \mathcal{A}^{(I)}_\lambda(s,z_s) = \frac{1}{2i} \bigg[ \mathcal{A}^{(I)}_\lambda(s + i\epsilon, z_s) - \mathcal{A}^{(I)}_\lambda(s-i\epsilon,z_s) \bigg],
  \end{equation}
%%
which we compute by assuming the reaction proceeds through a two pion intermediate state and integrating over the allowed two-body phase space, i.e.
%%
  \begin{equation}
    A(p_A) \pi(\overline{p}_3) \rightarrow \pi(q_1)\pi(q_2) \to \pi^\prime(p_1) \pi^\prime(p_2).
  \end{equation}
%%
Note the condition is on the discontinuity and not simply the imaginary part as the discontinuity is not purely imaginary in \(1 \to 3\) processes.

%#########################################################################################################
%#########################################################################################################
%#########################################################################################################
\subsection{Direct-Channel Contribution and Omn\`{e}s Solution}
The starting point for out KT equations is the unitarity relation for the helicity amplitude as in \cite{Danilkin:2014cra}
%%
  \begin{align}
    \label{eq:unitarity}
    \Disc \mathcal{A}^{(I)}_\lambda(p_A \overline{p}_3 \to p_1 p_2 ) =&\; \frac{\rho(s)}{64 \pi^2} \int d\Omega_s^\prime  \; {\mathcal{T}}^{(I)^*}(q_1q_2 \to p_1p_2) \times \mathcal{A}^{(I)}_\lambda(p_A \overline{p}_{3} \to q_1 q_2 ) \nonumber \\
%%
    =& \; \frac{\rho(s)}{64 \pi^2} \int d\Omega_s^\prime  \; {\mathcal{T}}^{(I)^*}(s,z_s^{\prime\prime}) \times \mathcal{A}^{(I)}_\lambda(s,z_s^{\prime})
  \end{align}
%%
where \({\tau^{(I)}}\) is the elastic \(\pi\pi\) scatting amplitude with definite isospin-\(I\), \(\rho(s) = \sqrt{1 - 4m_\pi^2/s}\) is the two body intermediate phase space, and the integration is over the angles \(\theta^\prime\) and \(\varphi^\prime\) of the intermediate state momenta. This intermediate frame is related to the initial \(A\pi\) scattering frame (determined by \(\theta\) and \(\varphi = 0\)) by an angle, \(\theta^{\prime\prime}\), given by (see eq. 6.71 in \cite{MS})
%%
  \begin{equation}
    \cos \theta^{\prime\prime} = \cos \theta \cos \theta^\prime + \cos \varphi^\prime \sin\theta \sin \theta^\prime.
  \end{equation}
%%

For the elastic pion scattering amplitude we use the standard partial wave decomposition (see eq. 16 in \cite{Danilkin:2014cra}), using \( z_s^{\prime\prime} = \cos \theta_s^{\prime\prime}\),
%%
  \begin{align}
    \label{eq:elastic-pion}
    \mathcal{T}^{(I)}(s, z_s^{\prime\prime}) =& \; 32 \, \pi \sum_{\ell=0  }^\infty \; (2\ell+1) \, P_{\ell}(z_s^{\prime\prime}) \; \tau_\ell^{(I)}(s) \nonumber \\
%%
    =& \; 128 \, \pi^2 \;
    \sum_{\ell=0}^\infty \sum_{m=-\ell}^{\ell}
    (2\ell +1 ) \; Y^m_\ell(\theta_s,0) \; {Y^m_\ell}^*(\theta_s^\prime, \varphi^\prime)
      \; \tau_\ell^{(I)}(s) \; .
  \end{align}
%%
Here the \(\pi\pi\) partial wave amplitudes are parameterized by the phase shifts, \(\delta\) and inelasticities \(\eta\) by:
%%
  \begin{equation}
    \tau^{(I)}_\ell(s) =
    \frac{
    \eta_\ell^{(I)}(s) \; e^{2 i \, \delta_\ell^{(I)}(s)} - 1
    }{
    2 \, i \; \rho(s) \, .
    }
  \end{equation}
%%
We may use the GKPY parameterizations \cite{Garcia-Martin2011} for the above up to \(\sim 2\) GeV and smoothly extrapolate to higher energies.

First we evaluate the left-hand side of \cref{eq:unitarity}. Because we assume our kinematic-singularity-free isobar functions, \(\hat{a}^{(I)}_{j \lambda}(s)\), only have a right-hand cut associated with unitarity the discontinuity comes only from the \(s\)-channel isobar:
%%
  \begin{equation}
    \label{eq:discontinuity}
    \Disc \mathcal{A}^{(I)}_\lamp = \sum_{j^\prime=0}^\jpmax \; (2 j^\prime +1)
    \; d^{j^\prime}_{\lamp 0}(\theta_s) \; \Disc a^{(I)}_{\jp \lamp}(s) \;
  \end{equation}
%%
Taking the \(j\)-th partial wave projection we get
%%
  \begin{equation}
    \label{eq:pw-disc}
    \frac{1}{2} \int_{-1}^1 dz_s  \; d^j_{\lambda 0}(\theta_s) \; \Disc \mathcal{A}^{(I)}_\lamp =
    \Disc a^{(I)}_{j\lambda}(s).
  \end{equation}
%%
Next we move on to the homogeneous direct channel contribution of the right-hand side of \cref{eq:unitarity}. Combining \cref{eq:isobar,eq:elastic-pion} (note we have to be careful about the kinematic singularities we removed from the \(d\)-function in \cref{eq:halfangle} and \(\varphi^\prime \not= 0\) in \cref{eq:helicity}),
 the \(s\)-channel part of the integrand in \cref{eq:unitarity} is given by:
%%
  \begin{align}
      \label{eq:direct-channel-angle}
     \int d\Omega_s^\prime \; {Y^m_\ell}^*(\theta_s^\prime, \varphi^\prime) \times d_{\lambda0}^{j^\prime}(\theta_s^\prime) \, e^{i\lambda \varphi^\prime} =&
     \; \sqrt{\frac{4\pi}{2j^\prime+1}\frac{(j^\prime+\lambda)!}{(j^\prime-\lambda)!}} \int d\Omega_s^\prime \; {Y^m_\ell}^*(\theta_s^\prime, \varphi^\prime) \;  Y^\lambda_{j^\prime}(\theta_s^\prime, \varphi^\prime) \nonumber \\
%%
    =& \;  \sqrt{\frac{4\pi}{2j^\prime+1}\frac{(j^\prime+\lambda)!}{(j^\prime-\lambda)!}} \;  \delta_{m\lambda} \; \delta_{\ell j^\prime}.
  \end{align}
%%
The entire direct-channel contribution then is (converting back from spherical hamronics with \(\phi = 0\) to \(d\)-functions):
%%
  \begin{align}
     2 \,  \sum_{\jp=0}^\jpmax  \,
       (2\jp +1) \; d_\lambda^\jp(\theta_s)
        \times
        \bigg[\rho(s) \; \tau^{(I)^*}_\jp(s) \; a^{(I)}_{\jp\lambda}(s) \bigg ]
        \; ,
  \end{align}
%%
Again taking the \(j\)-th partial wave as in \cref{eq:pw-disc}, we get the direct-channel contribution is simply
%%
  \begin{equation}
    \label{eq:pw-direct}
    \bigg[
    \rho(s) \; {\tau}^{(I)^*}_j(s) \; a^{(I)}_{j\lambda}(s)
    \bigg] \, .
  \end{equation}
%%
Comparing this to \cref{eq:pw-disc}, we see that if we ignore any contributions from the cross-channel we have the unitarity relation:
%%
  \begin{equation}
    \label{omnes-unitarity}
    \Disc \hat{a}^{(I)}_{j\lambda}(s) = \rho(s) \; {\tau}^{(I)^*}_j(s) \; \hat{a}_{j\lambda}^{(I)}(s)
  \end{equation}
%%
where we have additionally noticed the kinematic singularities are the same on the left- and right-hand sides and therefore cancel. If we use the fact that the isobar functions are analytic and have only a RHC in the complex plane and obey a dispersion relation,
%%
  \begin{equation}
    \hat{a}_{j\lambda}^{(I)}(s) = \frac{1}{2i} \int_\sth^\infty \frac{ds^\prime}{\pi}
    \; \frac{
    \Disc \hat{a}_{j\lambda}^{(I)}(s)
    }{
    s^\prime - s
    }
  \end{equation}
%%
we find this admits a solution of the Omn\`{e}s-Muskhelishvili form \cite{Omnes1958,Kamal1979,1953sie..book.....M}:
%%
  \begin{equation}
    \hat{a}_{j\lambda}^{(I)} = \text{exp} \;
    \bigg\{
    \frac{s}{\pi} \int_{\sth}^\infty ds^\prime
     \frac{
     \delta_{j}^{(I)}(s^\prime)
     }{
     s^\prime \, (s^\prime - s)
     }
    \bigg\}
  \end{equation}
%%
where \(\delta_j^{(I)}(s)\) is the \(j\)-th partial-wave, isospin-\(I\), \(\pi\pi\) phase-shift.
{\color{red} In general, how should we expect the helicity \(\lambda\) to appear in the Omnes solution? \Cref{omnes-unitarity} seems to suggest that, if we ignore cross-channel effects, \(\hat{a}^{(I)}_{j0}(s) = \hat{a}^{(I)}_{j+}(s)\) because they obey the exact same unitarity constraint.

Does helicity only factor into the angular dependence? Should it be :
\begin{equation}
    a^{(I)}_{j\lambda}(s) = K_\lambda \; (k(s)q(s))^{j-|\lambda|} \; \hat{a}^{(I)}_j(s),
\end{equation}
with the kinematic-singularity-free partial-wave \textit{of definite isospin} having no dependence on the helicity of the initial decay?
}
%#########################################################################################################
%#########################################################################################################
%#########################################################################################################
\subsection{Inhomogeneous Contribution and Cross-Channel Effects}
 A defining aspect of the Khuri-Trieman formulism is the inhomogeneous part of the dispersion relation which incorporates three-body effects in the final state. This part of the angular integral cannot be done analytically:
%%
  \begin{align}
    \label{eq:cross-0}
   \sum_\lamp \sum_\Ip \sum_{\jp=0}^\jpmax (2\jp + 1) \; \frac{1}{2} \, C_{II^\prime} \int d\Omega^\prime_s
    \; {Y^m_\ell}^*(\theta_s^\prime, \varphi^\prime)
    \times  d^J_{\lamp \lambda}(\hat{\theta}^\prime_1) \;
    \bigg[
     d_{\lamp 0}^{j^\prime}(z_t^\prime) \; e^{i\lambda \varphi^\prime} \; a^{(\Ip)}_{\jp \lamp}(t^\prime)
    \bigg]
  \end{align}
%%
where \(t^\prime = t(s,z_s^\prime)\), \(z_t^\prime = z_t(s,z_s^\prime)\), and \(\hat{\theta}_1^\prime = \hat{\theta}_1(s,z_s^\prime)\).

Instead we take the \(j\)-th partial-wave right away and examine the prefactors in \cref{eq:unitarity} involving the \(\theta_s\)-dependent spherical harmonic in \cref{eq:elastic-pion}:
%%
  \begin{align}
    \label{eq:cross-1}
    \sum_{\ell, m} \; (2\ell +1) \;
      \bigg[ \frac{1}{2}
       \int dz_s \;
      d_{\lambda 0}^j(\theta_s) \; Y^m_\ell(\theta_s,0)
       \bigg]
        \times \rho(s) \; {\tau}^{(I)^*}_\ell(s)
    %%
  = \frac{1}{2} (2\ell +1) \; \sqrt{\frac{4\pi}{2j+1}} \; \rho(s) \; \tau_j^{(I)*}(s) \; \delta_{\lambda m} \; \delta_{j \ell}.
  \end{align}
%%
Now we integrate \cref{eq:cross-0} over \(\varphi^\prime\),
%%
  \begin{align}
    \label{eq:cross-2}
    2\pi \; \sum_\lamp \sum_\Ip \sum_{\jp=0}^\jpmax (2\jp + 1) \; \frac{1}{2}C_{II^\prime} \; \sqrt{\frac{4\pi}{2\ell+1}} \;
     %%
     \int dz_s^\prime \; d_{m0}^{\,\ell}(\theta_s^\prime)
     \times
      d_{\lamp \lambda}^J(\hat{\theta}^\prime_1)
     \bigg[
      d_{\lamp 0}^{j^\prime}(z_t^\prime) \; a^{(\Ip)}_{j^\prime\lamp}(t^\prime)
     \bigg] \; .
  \end{align}
%%
Combining \cref{eq:cross-1,eq:cross-2}, we get the inhomogeneous contribution:
%%
  \begin{align}
      \label{eq:pw-cross}
     \rho(s) \; \tau^{(I)*}_j(s) \times
     \bigg[
      \sum_{\lamp \Ip} \frac{1}{2}C_{I\Ip} \; \sum_{\jp=0}^\jpmax (2\jp+1) \;
      \int d z_s^\prime \; d_{\lambda0}^j(\theta^\prime_s) \;
      d_{\lamp \lambda}^J(\hat{\theta}^\prime_1) \; d_{\lamp 0}^\jp(z_t^\prime) \; a^{(\Ip)}_{\jp \lamp}(t^\prime)
      \bigg] \; .
  \end{align}
%%

Finally, combining \cref{eq:pw-cross} with \cref{eq:pw-disc,eq:pw-direct} and removing the prime from the angle, we arrive at the KT equation for the decay process:
%%
  \begin{align}
    \label{kt-final}
    \Disc a_{j\lambda}^{(I)}(s) = \rho(s) \; \tau^{(I)*}_j(s) \times
    \bigg[
    a_{j\lambda}^{(I)}(s) +
    \sum_{\lamp \Ip \jp}  (2\jp+1) \; C_{I\Ip} \;
    \int d z_s \;  d_{\lambda0}^j(\theta_s) \; d_{\lamp \lambda}^J(\hat{\theta}_1) \;
       d_{\lamp 0}^\jp(z_t) \; a^{(\Ip)}_{\jp \lamp}(t)
    \bigg] \; .
  \end{align}
%%
To directly compare with \cite{Danilkin:2014cra,Niecknig:2012sj}, we define the kinematic-singularity-free derivatives of Legendre polynomials
%%
    \begin{equation}
    \tilde{P}_{j}^\lambda(s) = (k(s)q(s))^{j-\lambda} \, \frac{d^\lambda}{{dz_s}^\lambda} (P_j(z_s)) \; .
  \end{equation}
%%
With these, \cref{eq:k-factor,eq:kibble,eq:halfangle}, we may write \cref{kt-final} in terms of the kinematic-singularity-free isobar amplitudes:
%%
  \begin{align}
    \Disc \hat{a}_{j\lambda}^{(I)}(s) = \rho(s) \; \tau^{(I)*}_j(s) \;
    \bigg[
    \hat{a}_{j\lambda}^{(I)}(s) + \sum_{\jp \lamp \Ip}  &(2\jp+1) \; C_{I\Ip} \;
    \frac{(j-\lambda)!}{(j+\lambda)!} \nonumber \\
    &\times \int d z_s \; d_{\lamp \lambda}^J(\hat{\theta}_1) \;
    \bigg(\frac{\lambda_M(t)}{\lambda_M(s)}\bigg)^{J/2} \;
    \bigg(\frac{\phi}{4s}\bigg)^\lamp \;
    \frac{
    \tilde{P}_{j}^\lambda(s) \, \tilde{P}_{\jp}^\lamp(t)
    }{
    (k(s)q(s))^{2j}
    } \;
    \hat{a}_{\jp \lamp}^{(\Ip)}(t)
    \bigg] \; .
  \end{align}
%%
The ratio of \(\lambda_M\) arrise from the additional factors of momenta near threshold in \cref{eq:k-factor-tilde} with \(Y_i = J\). 

%#########################################################################################################
%#########################################################################################################
%#########################################################################################################
