%#########################################################################################################
%#########################################################################################################
%#########################################################################################################
\section{Khuri-Treiman Equations} \label{sec:unitarity}

Now we wish to impose impose elastic unitarity on each channel, via the KT equations.
Unitarity imposes a condition on the analytic structure in the complex \(s\)-plane. We assume that the isobar amplitudes only have a right-hand cut from \(s_\text{th} = 4m_\pi^2\) to \(\infty\) associated by the threshold opening of the \(\pi\pi\) final state. Unitarity tells us the discontinuity across this cut is
%%
  \begin{equation}
      \Disc \mathcal{A}^{(I)}_\lambda(s,z_s) = \frac{1}{2i} \bigg[ \mathcal{A}^{(I)}_\lambda(s + i\epsilon, z_s) - \mathcal{A}^{(I)}_\lambda(s-i\epsilon,z_s) \bigg],
  \end{equation}
%%
which we compute by assuming the reaction proceeds through a two pion intermediate state and integrating over the allowed two-body phase space, i.e.
%%
  \begin{equation}
    A(p_A) \pi(\overline{p}_3) \rightarrow \pi(q_1)\pi(q_2) \to \pi^\prime(p_1) \pi^\prime(p_2).
  \end{equation}
%%
Note the condition is on the discontinuity and not simply the imaginary part as the discontinuity is not purely imaginary in \(1 \to 3\) processes.

%#########################################################################################################
%#########################################################################################################
%#########################################################################################################
\subsection{Direct-Channel Contribution and Omn\`{e}s Solution}
The starting point for out KT equations is the unitarity relation for the helicity amplitude as in \cite{Danilkin:2014cra}
%%
  \begin{align}
    \label{eq:unitarity}
    \Disc \mathcal{A}^{(I)}_\lambda(p_A \overline{p}_3 \to p_1 p_2 ) =&\; \frac{\rho(s)}{64 \pi^2} \int d\Omega_s^\prime  \; {\mathcal{T}}^{(I)^*}(q_1q_2 \to p_1p_2) \times \mathcal{A}^{(I)}_\lambda(p_A \overline{p}_{3} \to q_1 q_2 ) \nonumber \\
%%
    =& \; \frac{\rho(s)}{64 \pi^2} \int d\Omega_s^\prime  \; {\mathcal{T}}^{(I)^*}(s,z_s^{\prime\prime}) \times \mathcal{A}^{(I)}_\lambda(s,z_s^{\prime})
  \end{align}
%%
where \({\tau^{(I)}}\) is the elastic \(\pi\pi\) scatting amplitude with definite isospin-\(I\), \(\rho(s) = \sqrt{1 - 4m_\pi^2/s}\) is the two body intermediate phase space, and the integration is over the angles \(\theta^\prime\) and \(\varphi^\prime\) of the intermediate state momenta. This intermediate frame is related to the initial \(A\pi\) scattering frame (determined by \(\theta\) and \(\varphi = 0\)) by an angle, \(\theta^{\prime\prime}\), given by (see eq. 6.71 in \cite{MS})
%%
  \begin{equation}
    \cos \theta^{\prime\prime} = \cos \theta \cos \theta^\prime + \cos \varphi^\prime \sin\theta \sin \theta^\prime.
  \end{equation}
%%

For the elastic pion scattering amplitude we use the standard partial wave decomposition (see eq. 16 in \cite{Danilkin:2014cra}), using \( z_s^{\prime\prime} = \cos \theta_s^{\prime\prime}\),
%%
  \begin{align}
    \label{eq:elastic-pion}
    \mathcal{T}^{(I)}(s, z_s^{\prime\prime}) =& \; 32 \, \pi \sum_{\ell=0  }^\infty \; (2\ell+1) \, P_{\ell}(z_s^{\prime\prime}) \; \tau_\ell^{(I)}(s) \nonumber \\
%%
    =& \; 128 \, \pi^2 \;
    \sum_{\ell=0}^\infty \sum_{m=-\ell}^{\ell}
     Y^m_\ell(\theta_s,0) \; {Y^m_\ell}^*(\theta_s^\prime, \varphi^\prime)
      \; \tau_\ell^{(I)}(s)
  \end{align}
%%
Here the \(\pi\pi\) partial wave amplitudes are parameterized by the phase shifts, \(\delta\) and inelasticities \(\eta\) by:
%%
  \begin{equation}
    \tau^{(I)}_\ell(s) =
    \frac{
    \eta_\ell^{(I)}(s) \; e^{2 i \, \delta_\ell^{(I)}(s)} - 1
    }{
    2 \, i \; \rho(s) \, .
    }
  \end{equation}
%%
We may use the GKPY parameterizations \cite{Garcia-Martin2011} for the above up to \(\sim 2\) GeV and smoothly extrapolate to higher energies.

First we evaluate the left-hand side of \cref{eq:unitarity}. Because we assume our kinematic-singularity-free isobar functions, \(\hat{a}^{(I)}_{j \lambda}(s)\), only have a right-hand cut associated with unitarity the discontinuity comes only from the \(s\)-channel isobar:
%%
  \begin{equation}
    \label{eq:discontinuity}
    \Disc \mathcal{A}^{(I)}_\lambda = \sum_{j^\prime=|\lamp|}^\jpmax \; (2 j^\prime +1)
    \; d^{j^\prime}_{\lambda 0}(\theta_s) \; \Disc a^{(I)}_{\jp \lambda}(s) \;
  \end{equation}
%%
Taking the \(j\)-th partial wave projection we get
%%
  \begin{equation}
    \label{eq:pw-disc}
   \int_{-1}^1 dz_s  \; d^j_{\lambda 0}(\theta_s) \; \Disc \mathcal{A}^{(I)}_\lambda =
    \frac{\delta_{j\jp}}{2j+1} \; \Disc a^{(I)}_{j\lambda}(s) \; .
  \end{equation}
%%

Next we move on to the homogeneous direct channel contribution of the right-hand side of \cref{eq:unitarity}, the \(s\)-channel part of the integrand in \cref{eq:unitarity} is given by:
%%
  \begin{align}
      \label{eq:direct-channel-angle}
     \int d\Omega_s^\prime \; {Y^m_\ell}^*(\theta_s^\prime, \varphi^\prime) \times d_{\lambda0}^{j^\prime}(\theta_s^\prime) \, e^{i\lambda \varphi^\prime} =&
     \;
     \sqrt{\frac{4\pi}{2j^\prime+1}}
      \int d\Omega_s^\prime \; {Y^m_\ell}^*(\theta_s^\prime, \varphi^\prime) \;  Y^\lambda_{j^\prime}(\theta_s^\prime, \varphi^\prime) \nonumber \\
%%
    =& \;  \sqrt{\frac{4\pi}{2j^\prime+1}} \;  \delta_{m\lambda} \; \delta_{\ell j^\prime}.
  \end{align}
%%
The entire direct-channel contribution then is (converting back from spherical hamronics with \(\phi = 0\) to \(d\)-functions):
%%
  \begin{align}
     2 \,  \sum_{\jp=|\lambda|}^\jpmax  \,
       (2\jp +1) \; d_\lambda^\jp(\theta_s)
        \times
        \bigg[\rho(s) \; \tau^{(I)^*}_\jp(s) \; a^{(I)}_{\jp\lambda}(s) \bigg ]
        \; ,
  \end{align}
%%
Again taking the \(j\)-th partial wave as in \cref{eq:pw-disc}, we get the direct-channel contribution is simply
%%
  \begin{equation}
    \label{eq:pw-direct}
    \bigg[
    \rho(s) \; {\tau}^{(I)^*}_j(s) \; a^{(I)}_{j\lambda}(s)
    \bigg] \, .
  \end{equation}
%%
Comparing this to \cref{eq:pw-disc}, we see that if we ignore any contributions from the cross-channel we have the unitarity relation:
%%
  \begin{equation}
    \label{omnes-unitarity}
    \Disc \hat{a}^{(I)}_{j\lambda}(s) = \rho(s) \; {\tau}^{(I)^*}_j(s) \; \hat{a}_{j\lambda}^{(I)}(s)
  \end{equation}
%%
where we have additionally noticed the kinematic singularities are the same on the left- and right-hand sides and therefore cancel. If we use the fact that the isobar functions are analytic and have only a RHC in the complex plane and obey a (in general subtracted) dispersion relation,
%%
  \begin{equation}
    \label{dispersion}
    \hat{a}_{j\lambda}^{(I)}(s) = \frac{1}{2i} \int_\sth^\infty \frac{ds^\prime}{\pi}
    \; \frac{
    \Disc \hat{a}_{j\lambda}^{(I)}(s)
    }{
    s^\prime - s
    }
  \end{equation}
%%
we find this admits a solution of the Omn\`{e}s-Muskhelishvili form \cite{Omnes1958,Kamal1979,1953sie..book.....M}:
%%
  \begin{equation}
    \hat{a}_{j\lambda}^{(I)} =
    F^{(I)}_{j\lambda}(s) \; \text{exp} \;
    \bigg\{
    \frac{s}{\pi} \int_{\sth}^\infty ds^\prime
     \frac{
     \delta_{j}^{(I)}(s^\prime)
     }{
     s^\prime \, (s^\prime - s)
     }
    \bigg\}
  \end{equation}
%%
where \(\delta_j^{(I)}(s)\) is the \(j\)-th partial-wave, isospin-\(I\), \(\pi\pi\) phase-shift and \(F\) is a helicity-dependent polynomial arrising from possible subtractions.
We note that the dependence on \(\lambda\) only appears in the subtractions used in the dispersion relation. If we consider unsubtracted dispersion relations as \cref{dispersion} then the Omn\`{e}s solution is entirely determined by the \(\pi\pi\) phase-shift and does not depend on the helicity of the initial decay.
%#########################################################################################################
%#########################################################################################################
%#########################################################################################################
\subsection{Inhomogeneous Contribution and Cross-Channel Effects}
 A defining aspect of the Khuri-Trieman formulism is the inhomogeneous part of the dispersion relation which incorporates three-body effects in the final state. This part of the angular integral cannot be done analytically:
%%
  \begin{align}
    \label{eq:cross-0}
   \sum_\lamp \sum_\Ip \sum_{\jp=|\lamp|}^\jpmax (2\jp + 1) \; \frac{1}{2} \, C_{II^\prime} \int d\Omega^\prime_s
    \; {Y^m_\ell}^*(\theta_s^\prime, \varphi^\prime)
    \times  d^J_{\lamp \lambda}(\hat{\theta}^\prime_1) \;
    \bigg[
     d_{\lamp 0}^{j^\prime}(z_t^\prime) \; e^{i\lambda \varphi^\prime} \; a^{(\Ip)}_{\jp \lamp}(t^\prime)
    \bigg]
  \end{align}
%%
where \(t^\prime = t(s,z_s^\prime)\), \(z_t^\prime = z_t(s,z_s^\prime)\), and \(\hat{\theta}_1^\prime = \hat{\theta}_1(s,z_s^\prime)\).

Instead we take the \(j\)-th partial-wave right away and examine the prefactors in \cref{eq:unitarity} involving the \(\theta_s\)-dependent spherical harmonic in \cref{eq:elastic-pion}:
%%
  \begin{align}
    \label{eq:cross-1}
    \sum_{\ell, m} \;
      \bigg[
       \int dz_s \;
      d_{\lambda 0}^j(\theta_s) \; Y^m_\ell(\theta_s,0)
       \bigg]
        \times \rho(s) \; {\tau}^{(I)^*}_\ell(s)
    %%
  = \sqrt{\frac{2j+1}{4\pi}} \; \rho(s) \; \tau_j^{(I)*}(s) \; \delta_{\lambda m} \; \delta_{j \ell}.
  \end{align}
%%
Now we integrate \cref{eq:cross-0} over \(\varphi^\prime\),
%%
  \begin{align}
    \label{eq:cross-2}
    4\pi \; \sum_\lamp \sum_\Ip \sum_{\jp=|\lamp|}^\jpmax  \frac{1}{2}C_{II^\prime} \; \sqrt{\frac{2\ell+1}{4\pi}} \;
     %%
     \int dz_s^\prime \; d_{m0}^{\,\ell}(\theta_s^\prime)
     \times
      d_{\lamp \lambda}^J(\hat{\theta}^\prime_1)
     \bigg[
      d_{\lamp 0}^{j^\prime}(z_t^\prime) \; a^{(\Ip)}_{j^\prime\lamp}(t^\prime)
     \bigg] \; .
  \end{align}
%%
Combining \cref{eq:cross-1,eq:cross-2}, we get the inhomogeneous contribution:
%%
  \begin{align}
      \label{eq:pw-cross}
     \rho(s) \; \tau^{(I)*}_j(s) \times
     \bigg[
      \sum_{\lamp \Ip} \; \frac{1}{2}C_{I\Ip} \; \sum_{\jp=|\lamp|}^\jpmax \;
      \int d z_s^\prime \; d_{\lambda0}^j(\theta^\prime_s) \;
      d_{\lamp \lambda}^J(\hat{\theta}^\prime_1) \; d_{\lamp 0}^\jp(z_t^\prime) \; a^{(\Ip)}_{\jp \lamp}(t^\prime)
      \bigg] \; .
  \end{align}
%%

Finally, combining \cref{eq:pw-cross} with \cref{eq:pw-disc,eq:pw-direct} and removing the prime from the angle, we arrive at the KT equation for the decay process:
%%
  \begin{align}
    \label{kt-final}
    \Disc a_{j\lambda}^{(I)}(s) = \rho(s) \; \tau^{(I)^*}_j(s) \times
    \bigg[
    a_{j\lambda}^{(I)}(s) +
    \sum_{\lamp \Ip \jp} \frac{1}{2}C_{I\Ip}
    \int d z_s \;  d_{\lambda0}^j(\theta_s) \; d_{\lamp \lambda}^J(\hat{\theta}_1) \;
       d_{\lamp 0}^\jp(z_t) \; a^{(\Ip)}_{\jp \lamp}(t)
    \bigg] \; .
  \end{align}
%%
We note that \cref{kt-final} is in terms of the isobar functions with kinematic singularities, thus we may define the kinematic integration kernal which is a ratio of \(K\)-functions \cref{eq:k-factor}:
%%
  \begin{equation}
    K_{\lamp\lambda}(t,s) = \frac{K_\lamp(t)}{K_\lambda(s)}
  \end{equation}
%%
to rewrite \cref{kt-final} in terms of the kinematic-singularity-free \(d\)-functions and isobar amplitudes:
%%
  \begin{align}
    \label{kt-final-KSF}
    \Disc \hat{a}_{j\lambda}^{(I)}(s) = \rho(s) \; \tau^{(I)^*}_j(s) \times
    \bigg[
    \hat{a}_{j\lambda}^{(I)}(s) +
    \sum_{\lamp \Ip \jp} \frac{1}{2}C_{I\Ip}
    \int d z_s \; K_{\lamp\lambda}(t,s) \; d_{\lambda0}^j(\theta_s) \; d_{\lamp \lambda}^J(\hat{\theta}_1) \;
       d_{\lamp 0}^\jp(z_t) \; \hat{a}^{(\Ip)}_{\jp \lamp}(t)
    \bigg] \; .
  \end{align}
%%
%#########################################################################################################
%#########################################################################################################
%#########################################################################################################
\subsection{Path of Integration}

%#########################################################################################################
%#########################################################################################################
%#########################################################################################################
