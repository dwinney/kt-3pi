%#########################################################################################################
%#########################################################################################################
%#########################################################################################################
\section{Khuri-Treiman Equations} \label{sec:unitarity}

Now we wish to impose impose elastic unitarity on each channel, via the KT equations.
Unitarity imposes a condition on the analytic structure in the complex \(s\)-plane. We assume that the isobar amplitudes only have a right-hand cut from \(s_\text{th} = 4m_\pi^2\) to \(\infty\) associated by the threshold opening of the \(\pi\pi\) final state. Unitarity tells us the discontinuity across this cut is
%%
  \begin{equation}
      \Disc \mathcal{A}^{(I)}_\lambda(s,z_s) = \frac{1}{2i} \bigg[ \mathcal{A}^{(I)}_\lambda(s + i\epsilon, z_s) - \mathcal{A}^{(I)}_\lambda(s-i\epsilon,z_s) \bigg],
  \end{equation}
%%
which we compute by assuming the reaction proceeds through a two pion intermediate state and integrating over the allowed two-body phase space, i.e.
%%
  \begin{equation}
    A(p_A) \pi(\overline{p}_3) \rightarrow \pi(q_1)\pi(q_2) \to \pi(p_1) \pi(p_2).
  \end{equation}
%%
Note the condition is on the discontinuity and not simply the imaginary part as the discontinuity is not purely imaginary in \(1 \to 3\) processes.

The starting point for out KT equations is the unitarity relation for the helicity amplitude as in \cite{Danilkin:2014cra}
%%
  \begin{align}
    \label{eq:unitarity}
    \Disc \mathcal{A}^{(I)}_\lambda(p_A \overline{p}_3 \to p_1 p_2 ) =&\; \frac{\rho(s)}{64 \pi^2} \int d\Omega_s^\prime  \; {\mathcal{T}}^{(I)^*}(q_1q_2 \to p_1p_2) \times \mathcal{A}^{(I)}_\lambda(p_A \overline{p}_{3} \to q_1 q_2 ) \nonumber \\
%%
    =& \; \frac{\rho(s)}{64 \pi^2} \int d\Omega_s^\prime  \; {\mathcal{T}}^{(I)^*}(s,z_s^{\prime\prime}) \times \mathcal{A}^{(I)}_\lambda(s,z_s^{\prime})
  \end{align}
%%
where \({\tau^{(I)}}\) is the elastic \(\pi\pi\) scatting amplitude with definite isospin-\(I\), \(\rho(s) = \sqrt{1 - 4m_\pi^2/s}\) is the two body intermediate phase space, and the integration is over the angles \(\theta^\prime\) and \(\varphi^\prime\) of the intermediate state momenta. This intermediate frame is related to the initial \(A\pi\) scattering frame (determined by \(\theta\) and \(\varphi = 0\)) by an angle, \(\theta^{\prime\prime}\), given by (see eq. 6.71 in \cite{MS})
%%
  \begin{equation}
    \cos \theta^{\prime\prime} = \cos \theta \cos \theta^\prime + \cos \varphi^\prime \sin\theta \sin \theta^\prime.
  \end{equation}
%%

For the elastic pion scattering amplitude we use the standard partial wave decomposition (see eq. 16 in \cite{Danilkin:2014cra}), using \( z_s^{\prime\prime} = \cos \theta_s^{\prime\prime}\),
%%
  \begin{align}
    \label{eq:elastic-pion}
    \mathcal{T}^{(I)}(s, z_s^{\prime\prime}) =& \; 32 \, \pi \sum_{\ell=0  }^\infty \; (2\ell+1) \, P_{\ell}(z_s^{\prime\prime}) \; \tau_\ell^{(I)}(s) \nonumber \\
%%
    =& \; 128 \, \pi^2 \; \sum_{\ell=0}^\infty \sum_{m=-\ell}^{\ell} (2\ell +1 ) Y^m_\ell(\theta_s,0) \; {Y^m_\ell}^*(\theta_s^\prime, \varphi^\prime) \; \tau_\ell^{(I)}(s) \; .
  \end{align}
%%

First we evaluate the left-hand side of \cref{eq:unitarity}. Because we assume our kinematic-singularity-free isobar functions, \(\hat{a}^{(I)}_{j \lambda}(s)\), only have a right-hand cut associated with unitarity the discontinuity comes only from the \(s\)-channel isobar:
%%
  \begin{equation}
    \label{eq:discontinuity}
    \Disc \mathcal{A}^{(I)}_\lamp = s^{\lamp/2}\, \sum_{j^\prime=0}^\jpmax \; (2 j^\prime +1) \; (k(s)q(s))^{j^\prime-\lamp}  \; \xi_{\lamp 0}(z_s)
    \; \hat{d}^{j^\prime}_{\lamp 0}(\theta_s) \; K^{j^\prime}_{\lamp0}(s) \; \Disc \hat{a}^{(I)}_{\jp \lamp}(s).
  \end{equation}
%%
Taking the \(j\)-th partial wave projection we get
%%
  \begin{equation}
    \label{eq:pw-disc}
    \frac{1}{2} \int_{-1}^1 dz_s \; P^\lambda_{j}(z_s) \; \Disc \mathcal{A}^{(I)}_\lamp =
    \bigg[ s^{\lambda/2} \, K^j_{\lambda 0}(s) \, (k(s)q(s))^{j-\lambda} \; \frac{(j+\lambda)!}{(j-
    \lambda)!} \bigg] \times \Disc \hat{a}^{(I)}_{j\lambda}(s).
  \end{equation}
%%

Next we move on to the homogeneous direct channel contribution of the right-hand side of \cref{eq:unitarity}. Combining \cref{eq:isobar,eq:elastic-pion} (note we have to be careful about the kinematic singularities we removed from the \(d\)-function in \cref{eq:halfangle} and \(\varphi^\prime \not= 0\) in \cref{eq:helicity}),
 the \(s\)-channel part of the integrand in\cref{eq:unitarity} is given by:
%%
  \begin{align}
      \label{eq:direct-channel-angle}
     \int d\Omega_s^\prime \; {Y^m_\ell}^*(\theta_s^\prime, \varphi^\prime) \times \xi_{\lambda0}(z_s^\prime) \, \hat{d}_{\lambda0}^{j^\prime}(\theta_s^\prime) \, e^{i\lambda \varphi^\prime} =&
     \; \sqrt{\frac{4\pi}{2j^\prime+1}\frac{(j^\prime+\lambda)!}{(j^\prime-\lambda)!}} \int d\Omega_s^\prime \; {Y^m_\ell}^*(\theta_s^\prime, \varphi^\prime) \;  Y^\lambda_{j^\prime}(\theta_s^\prime, \varphi^\prime) \nonumber \\
%%
    =& \;  \sqrt{\frac{4\pi}{2j^\prime+1}\frac{(j^\prime+\lambda)!}{(j^\prime-\lambda)!}} \;  \delta_{m\lambda} \; \delta_{\ell j^\prime}.
  \end{align}
%%
The entire direct-channel contribution then is
%%
  \begin{align}
    2 \; \sum_{j^\prime=0}^\jpmax \sqrt{\frac{4\pi}{2j^\prime+1}\frac{(j^\prime+\lambda)!}{(j^\prime-\lambda)!}}& \, {Y^\lambda_{j^\prime}}(\theta_s,0) \; s^{\lambda/2} \,  K^{j^\prime}_{\lambda0}(s)
    \; (2j^\prime +1) \; (k(s)q(s))^{j^\prime - \lambda}
    \bigg[\rho(s) \; \tau^{(I)^*}_j(s) \; \hat{a}^{(I)}_{j\lambda}(s) \bigg ] \nonumber \\
    =& \;  2\;  \sum_{j^\prime=0}^\jpmax \; P^\lambda_j(z_s) \; s^{\lambda/2} \, K^{j^\prime}_{\lambda0}(s)  \; (2j^\prime +1) \; (k(s)q(s))^{j^\prime - \lambda}
    \bigg[\rho(s) \; {\tau}^{(I)^*}_{j^\prime}(s) \; \hat{a}^{(I)}_{\jp\lambda}(s) \bigg ],
  \end{align}
%%
and again taking the \(j\)-th partial wave as in eq.~\ref{eq:pw-disc}, we get the direct-channel contribution:
%%
  \begin{equation}
    \label{eq:pw-direct}
   \bigg[ s^{\lambda/2} \, K^j_{\lambda 0}(s) \; (k(s)q(s))^{j - \lambda} \; \frac{(j+\lambda)!}{(j- \lambda)!} \bigg] \times \rho(s) \; {\tau}^{(I)^*}_j(s) \; \hat{a}^{(I)}_{j\lambda}(s).
  \end{equation}
%%

Finally, in the cross-channels we have the inhomogeneous part of the dispersion relation. The angular integral cannot be done analytically:
%%
  \begin{align}
   \sum_{\emp \;\Ip} \sum_{\jp=0}^\jpmax (2\jp + 1) \; C_{II^\prime} \int d\Omega^\prime_s
    \; d^J_{\lambda \emp}(\hat{\theta}^\prime_1) \; &{Y^m_\ell}^*(\theta_s^\prime, \varphi^\prime)  \; e^{i\lambda \varphi^\prime}
     \nonumber \\
    &\times \bigg[
    {t^\prime}^{\emp/2} \; \xi_{\emp 0}(z_t^\prime) \; (k(t^\prime)q(t^\prime))^{j^\prime-\emp}
    K^{j^\prime}_{\emp0}(t^\prime) \; \hat{d}_{\emp 0}^{j^\prime}(z_t^\prime) \; \hat{a}^{(\Ip)}_{j^\prime\emp}(t^\prime)
    \bigg]
  \end{align}
%%
where \(t^\prime = t^\prime(s,z_s^\prime)\) and \(z_t^\prime = z_t^\prime(s,z_s^\prime)\).
Integrating over \(\varphi^\prime\),
%%
  \begin{align}
    \label{eq:cross-1}
     \sum_{\emp \;\Ip} \sum_{\jp=0}^\jpmax (2\jp + 1) \; C_{II^\prime} \; &\sqrt{\frac{2\ell+1}{4\pi}\frac{(\ell-\emp)!}{(\ell+\emp)!}} \; \int dz_s^\prime \; P_\ell^m(z_s^\prime)
     \; d_{\lambda \emp}^J(\hat{\theta}^\prime_1) \nonumber \\
     &\times \bigg[
     {t^\prime}^{\emp/2} \; \xi_{\emp 0}(z_t^\prime) \; (k(t^\prime)q(t^\prime))^{j^\prime-\emp}
     K^{j^\prime}_{\emp0}(t^\prime) \; \hat{d}_{\emp 0}^{j^\prime}(z_t^\prime) \; \hat{a}^{(\Ip)}_{j^\prime\emp}(t^\prime)
     \bigg] \; ,
  \end{align}
%%
we can take the \(j\)-th partial wave right away. Considering only the prefactors in front of the angular integral of \cref{eq:cross-1} in \cref{eq:unitarity} coming from the second spherical harmonic in \cref{eq:elastic-pion}:
%%
  \begin{align}
    \label{eq:cross-2}
     (2\ell+1)  \; \sum_{\ell, m} \bigg [ \int \frac{dz_s}{2} &\; P^\lambda_j(z_s) \; Y^m_\ell(\theta_s,0) \bigg] \; \times \; \rho(s) \; t_\ell^*(s)
    %%
  = \; \frac{(2j+1)}{2} \; \sqrt{\frac{4\pi}{2j+1} \frac{(j+\lambda)!}{(j-\lambda)!}} \;\rho(s) \; \tau_j^{(I)*}(s) \; \delta_{\lambda m} \; \delta_{j \ell}.
  \end{align}
%%
Combining \cref{eq:cross-1,eq:cross-2}, we get the inhomogeneous contribution:
%%
  \begin{align}
      \label{eq:pw-cross}
      (2j+1) \; \rho(s) \; &\tau^{(I)*}_j(s) \; \bigg[
      \sum_{\jp \emp \Ip} (2\jp+1) \; C_{I\Ip} \;  \nonumber \\
      &\times \int d z_s^\prime \; P_j^\lambda(z_s^\prime) \times
      d_{\lambda \emp}^J(\hat{\theta}^\prime_1) \;
      {t^\prime}^{\emp/2} \; \xi_{\emp 0}(z_t^\prime) \; (k(t^\prime)q(t^\prime))^{j^\prime-\emp}
      K^{j^\prime}_{\emp0}(t^\prime) \; \hat{d}_{\emp 0}^{j^\prime}(z_t^\prime) \; \hat{a}^{(\Ip)}_{j^\prime\emp}(t^\prime)
      \bigg]
  \end{align}
%%
which we can rewrite
in terms of derivatives of Legendre polynomials and the Lorentz-covariant Kibble function, \(\phi\), by defining (with \(\lambda \geq 0\)):
%%
    \begin{equation}
    \tilde{P}_{j}^\lambda(s) = (k(s)q(s))^{j-\lambda} \, \frac{d^\lambda}{{dz_s}^\lambda} (P_j(z_s))
    \mand
    \phi = (2 \sqrt{s} \; \sqrt{1-z_s^2} \;k(s)q(s))^2 \; .
  \end{equation}
%%
With these and removing the prime on the angle, we can rewrite a form analogous to \cite{Danilkin:2014cra}
%%
  \begin{equation}
    (2j+1) \; \rho(s) \; \tau^{(I)*}_j(s) \; \bigg[
    \sum_{\jp \emp \Ip} (2\jp+1) \; C_{I\Ip} \; \int dz_s  \;
    d_{\lambda \emp}^J(\hat{\theta}^\prime_1) \times
    \bigg(  \frac{\phi}{4 \, s}\bigg)^\emp \;
    \frac{\tilde{P}^\lambda_j(s) \, \tilde{P}_{j^\prime}^\emp(t)}{(k(s)q(s))^{2j}} \;  \; \hat{a}^{(\Ip)}_{j^\prime \emp}(t)
    \bigg]
  \end{equation}
%%
Thus combining \cref{eq:pw-disc,eq:pw-direct,eq:pw-cross} we arrive at the KT equations for the kinematic singularity-free helicity amplitudes:
%%
  \begin{align}
    &\Disc \hat{a}^j_\lambda(s) = \rho(s) \; t^*_{j}(s) \; \bigg[ \; \hat{a}^j_\lambda(s) \;+ \;  2\, (2j+1) \, \frac{(j-\lambda)!}{(j+\lambda)!} \nonumber \\
    & \times \; \bigg[
    \sum_{\jp \emp \Ip} (2\jp+1) \; \frac{C_{I\Ip}}{K_\lambda(s)}  \int d z_s^\prime \; P_j^\lambda(z_s^\prime) \times
    d_{\lambda \emp}^J(\hat{\theta}^\prime_1) \;
  K_{m^\prime}(t^\prime) \; (k(t^\prime)q(t^\prime))^{j^\prime-\emp}
  \hat{d}_{\emp 0}^{j^\prime}(z_t^\prime) \; \hat{a}^{(\Ip)}_{j^\prime\emp}(t^\prime)
    \bigg]
  \end{align}
%%
where we recall here, \(\xi_{\lambda 0}(z)\) and \(\hat{d}_{\lambda 0}^j(z)\) are given by eq.~\ref{eq:halfangle}.

We can rewrite eq.~\ref{eq:helicity-kt}
Then we write the KT equation for the \(j\)-th partial wave of a particle with helicity \(\lambda\) as:
%%
\begin{align}
  \label{eq:helicity-kt}
  \Disc \hat{a}^j_\lambda(s) = \rho(s) \; t^*_{j}(s) \, \bigg[ \; \hat{a}^j_\lambda(s) + (2j+1) \, \frac{(j-\lambda)!}{(j+\lambda)!} \sum_{j^\prime = 0}^\jpmax \, (2j^\prime+1)
  \int dz_s\;
  \bigg(  \frac{\phi}{4 \, s}\bigg)^\lambda \frac{\tilde{P}^\lambda_j(s) \, \tilde{P}_{j^\prime}^\lambda(t)}{(k(s)q(s))^{2j}} \;  \; \hat{a}^{j^\prime}_\lambda(t)\bigg]
\end{align}
%%
where we've defined the ratio of kinetic factors
%%
  \begin{equation}
    R^{jj^\prime}_{\lambda0}(s,z_s) = \frac{{t}^{\lambda/2}
    \; K^{j^\prime}_{\lambda0}(t) \; \xi_{\lambda0}(z_t)}{s^{\lambda/2} \; K^j_{\lambda0}(s) \;\xi_{\lambda0}(z_s)},
  \end{equation}
%%
with \(t = t(s,z_s)\) and \(z_t = z_t(s,z_s)\). We note that if the decaying particles has natural parity, i.e. \(P = (-1)^J\) such as the \(\phi/\omega\), then the kinematic factors are directly related to the Kibble function and are invariant between frames. Then we have \(R^{jj}_{\lambda0}(s,z_s) = 1\) and we recover eq. 18 in \cite{Danilkin:2014cra} for \(j=1\). We must also take care to remember that the kinematic functions \(K^j_{\lambda0}(x)\) may have different dependences near threshold for \(j < J\).
%#########################################################################################################
%#########################################################################################################
%#########################################################################################################
