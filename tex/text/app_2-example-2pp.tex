%#########################################################################################################
%#########################################################################################################
%#########################################################################################################
\section{The Tensor Isoscalar Decay (\(I^G \,J^{PC} = 0^- \, 2^{++}\))} \label{app:2pp}
%#########################################################################################################
In this section we consider the illustrative example of a tensor meson with natural parity, the \(f_2(1270)\), decaying into three pions with the formalism developed in this note.

First from \cref{parity-jpc}, we immediately have \(\hat{A}_0(s,z_s) = 0\) and we only have to consider \(\lambda = 1, \; 2\). Now we use \cref{eq:k-factor} to identify the kinematic singularities:
%%
  \begin{align}
    \label{2pp-k-factor}
  K_1(s) =  \frac{\sin\theta_s \;\lambda^{1/2}_\pi(s)}{ \sqrt{s} \; \lambda^{1/2}_M(s) }
  \mand
  K_2(s) = \frac{\sin^2\theta_s \;\lambda_\pi(s)}{s} \; .
  \end{align}
%%
This gives us:
%%
  \begin{align}
    \label{eq:2pp-1}
    \mathcal{A}_1(s,z_s) = s^{-1/2} \; \sin\theta_s \; \lambda^{-1/2}_M(s)  \;\lambda^{1/2}_\pi(s)\; \sum_{j = 1}^\infty \; (2j+1) \; (k(s)q(s))^{j-1} \; \hat{d}_{10}^j(\theta_s) \; \hat{A}_{j1}(s)
  \end{align}
%5
and
%%
  \begin{align}
    \label{eq:2pp-2}
    \mathcal{A}_2(s,z_s) = s^{-1} \; \sin^2\theta_s \; \lambda_\pi(s) \; \sum_{j=1}^\infty \; (2j+1) \; (k(s)q(s))^{j-2} \; \hat{d}_{20}^j(\theta_s) \; \hat{A}_{j2}(s) \; .
  \end{align}
%%
We may easily list off the only nontrivial kinematic constraint at threshold corresponds to \(\tau = 1\) in \cref{trans-constraint}:
%%
  \begin{equation}
    \bigg[
    \mathcal{A}_{1}(s,t,u) + i \; \mathcal{A}_{2}(s,t,u)
    \bigg] \sim \lambda_M^{1/2}(s) \;
  \end{equation}
%%
which, using \cref{2pp-k-factor}, see is very similar to \cref{constraint-invariant}:
%%
  \begin{equation}
    \bigg[
    \hat{d}_{10}^j \; \hat{A}_{j1}(s) +  (M\pm m_\pi) \; z_s \; \hat{d}_{20}^j(\theta_s) \; \hat{A}_{j2}
    \bigg] \to 0 \text{ as } \lambda_M(s)
  \end{equation}
%%
%# ########################################################################################################
%#########################################################################################################
%#########################################################################################################
